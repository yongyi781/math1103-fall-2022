\documentclass[11pt,oneside]{amsart}
\usepackage{geometry}
\usepackage{amssymb,parskip,mathtools}
\usepackage[shortlabels]{enumitem}
\usepackage[colorlinks]{hyperref}

\theoremstyle{definition}
\newtheorem{problem}{Problem}

\theoremstyle{plain}
\newtheorem{theorem}{Theorem}

\newcommand{\bC}{\mathbb{C}}
\newcommand{\bQ}{\mathbb{Q}}
\newcommand{\bR}{\mathbb{R}}
\newcommand{\bZ}{\mathbb{Z}}
\newcommand{\bE}{\mathbb{E}}
\newcommand{\eps}{\varepsilon}
\newcommand{\blank}{\underline{\hspace{1cm}}}
\newcommand{\longblank}{\underline{\hspace{2cm}}}

\DeclareMathOperator{\Var}{Var}

\title{MATH1103 Fall 2022\\
Problem Set 10}

\begin{document}
\maketitle
This problem set is due on Wednesday, November 16 at 11:59 pm. Each problem part is worth 3 points. Collaboration is encouraged. In all cases, you must write your own solutions, and and you must cite collaborators and resources used.

\begin{problem}
Comparison and/or ratio test practice. Determine whether each of the following series converges. General tip: Think about comparison before thinking about ratio test. Of course, think about both strategies in case one of them doesn't seem to be leading anywhere useful.
\begin{enumerate}[(a)]
  \item $\displaystyle\sum_{k=1}^\infty{\frac{10^k}{7+5^k}}$.
  \item $\displaystyle\sum_{k=1}^\infty k\cdot 3^{-k}$.
  \item $\displaystyle\sum_{k=1}^\infty\frac{\log k}k$.
  \item $\displaystyle\sum_{n=2}^\infty\frac 1{\sqrt{n^2-1}}$.
  \item $\displaystyle\sum_{n=1}^\infty\frac 1{n^n}$.
\end{enumerate}
\end{problem}

\begin{problem}
Partial fractions.
\leavevmode\begin{enumerate}[(a)]
  \item What is $\frac 12-\frac 13$? What is $\frac 13-\frac 14$? What is $\frac 14-\frac 15$? Make a conjecture based on your findings, then prove it.
  \item Using what you proved in part (a), find the sum
        \[\frac 1{1\cdot 2}+\frac 1{2\cdot3 }+\frac 1{3\cdot 4}+\frac 1{4\cdot 5}+\cdots+\frac 1{99\cdot 100}\]
        and the sum
        \[\frac 1{1\cdot 2}+\frac 1{2\cdot3 }+\frac 1{3\cdot 4}+\cdots.\]
  \item Challenge! Use your thinking skills, reflecting on how you solved the previous part, to find the sum
        \[\frac1{1\cdot 4}+\frac 1{2\cdot 5}+\frac 1{3\cdot 6}+\cdots.\]
  \item A slight change can make a problem much much harder. Let's now look at the following sum:
        \[\frac 1{1\cdot 2}+\frac 1{3\cdot 4}+\frac 1{5\cdot 6}+\dots.\]
        This sum is similar in form to the one in part (b) but the limit is now irrational! What does the internet (e.g. Wolfram Alpha) say this sum equals? (You might want to figure out how to express it as a summation so you can input it into the service.)

        Then give a guess as to how one might prove it. (Hint: the sum in the Zax problem of 2 psets ago converged to $1-\ln 2$. Maybe there's a connection\ldots)
  \item Prove that $\displaystyle\sum_{n=1}^\infty \frac 1{n^2}$ converges, using comparison and the result of part (b).\footnote{Finding the sum was a famous problem, called the
          Basel Problem because the Bernoulli family and Euler (all from Basel, Switzerland) worked
          on it. It was Euler who found the sum in 1734. We may see later how he did it.}
\end{enumerate}
\end{problem}

\begin{problem}
Around 1910, the Indian mathematician Srinivasa Ramanujan discovered that
\[
  \frac{2\sqrt{2}}{9801}\sum_{n=0}^\infty\frac{(4n)!(1103+26390n)}{(n!)^4396^{4n}}=\frac{1}{\pi}.
\]
Prove the more modest assertion, that the series converges at all.

\end{problem}

\begin{problem}
Rate the difficulty of each problem (1a, 1b, 1c, 1d, 1e, 2a, 2b, 2c, 2d, 2e, 3) according to the following scale. Your ratings will collectively let me know which areas are difficult in this class. Thanks for your feedback!
\begin{itemize}
  \item 1 -- Super easy, barely an inconvenience!
  \item 2 -- Not easy, but I was able to solve the problem on my own by comparing it with an example from class or the textbook.
  \item 3 -- Not easy, but I was able to solve the problem on my own through observations, analysis, and/or creative reasoning.
  \item 4 -- I made some progress but got stuck, and with help, I was able to solve the problem. I feel like I understand it now.
  \item 5 -- I could not start this problem without help, but after getting help I was able to solve the problem. I feel like I understand it now.
  \item 6 -- I could not start this problem without help, but after getting help I was able to solve the problem. However, I still don't feel like I understand what is going on in this problem.
  \item 7 -- I could not solve the problem, even with help.
\end{itemize}
\end{problem}
\end{document}
