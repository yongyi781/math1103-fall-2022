\documentclass[12pt, oneside]{article}   	% use "amsart" instead of "article" for AMSLaTeX format
\usepackage{geometry}                		% See geometry.pdf to learn the layout options. There are lots.
\geometry{letterpaper}                   		% ... or a4paper or a5paper or ... 
%\geometry{landscape}                		% Activate for for rotated page geometry
%\usepackage[parfill]{parskip}    		% Activate to begin paragraphs with an empty line rather than an indent
\usepackage{graphicx}				% Use pdf, png, jpg, or eps§ with pdflatex; use eps in DVI mode
\usepackage{amsmath}								% TeX will automatically convert eps --> pdf in pdflatex		
\usepackage{amssymb}

\title{}
\author{MATH1103   \ \ \ \ \ \ Final Exam A \ \ \ \  December 18, 2021 \\  \\ Closed book, closed notes; show all your work.}
\date{}							% Activate to display a given date or no date

\begin{document}
\maketitle
%\section{}
%\subsection{}
\begin{enumerate}



%1
\item (10 pts) For each of the following, tell if it's True or False; \emph{no reasons need to be given.}

$$(a) \ \dfrac{d}{dx} \ \int_{\sin(x)}^{0} e^{t^2} \, dt= -e^{(\sin(x))^2}. \ \ \ \ \ \ \ \  (b) \ \dfrac{d}{dx} \int_0^{\pi/2} \sin^2(x) \, dx=1.  \ \ \ \ \ \ \ \ \ (c) \ \sum_{n=1}^{\infty}  n^{-\cos(2)} \textrm{ diverges.}$$
$$  (d) \ \sum_{n=0}^{\infty} \dfrac{(-1)^n}{n!} = \dfrac{1}{e}. \ \ \ \ \ \ \  (e) \textrm{ If } a_n>0 \textrm{ and } \sum_{n=1}^{\infty} a_n \textrm{ converges, then } \sum_{n=1}^{\infty} (-1)^n a_n \textrm{ converges.}$$




%2 
\item (10 pts)  Find the following indefinite integrals, showing all your work.
$$(a) \ \int \dfrac{1}{1+ \sqrt{x}} \, dx \ \ \ \ \ \ \ \ \ \ \ (b) \ \int x^{3/2}\ln(4x) \, dx $$


%3
\item (10 pts)  Find the following indefinite integrals, showing all your work.

$$ (a) \ \int \, \dfrac{8x^2+9x+2}{x^3+2x^2} \, dx \ \ \ \ \ \ \ \ \ \ \ (b) \ \int \sin^6(3x) \cos^5(3x) \, dx$$

\

%4
\item   (10 pts) Consider the region R in the first quadrant bounded by the curve $y=e^{3x}$, the line $y=1$, and the line $x=1$.  This region is revolved around the $x$-axis to form a solid S.

\begin{enumerate}
\item Set up, \emph{but do not evaluate} the volume of S using the disk/washer method.



\item Set up, \emph{but do not evaluate} the volume of S using the shell method.

\item Using one of these, find the volume.
\end{enumerate}


%5
\item (20 pts) Determine if the following series converge or diverge. Justify your answers. If they converge, you do not have to find the values. 

$$(a)  \ \sum_{n=1}^{\infty} \dfrac{2^{4n}}{(n+1)!}  \ \ \ \ \ \ \ \ \ \ \ (b) \ \sum_{n=1}^\infty \dfrac{e+\cos^2(n)}{n^2}  \ \ \ \ \ \ \ \ \ \ \ (c) \ \sum_{n=7}^{\infty} \dfrac{(4n+5)^2}{3n^3}  \ \ \ \ \ \ \ \ \ \ \ (d) \ \sum_{n=1}^{\infty} n \left(\dfrac{3n+2}{2n} \right)^n$$

\

%7
\item (10 pts)  Determine if the following series diverge, converge absolutely, or converge conditionally, showing your work.
$$(a)  \ \sum_{n=4}^{\infty}  \dfrac{(-1)^{n+1}}{4n+3}  \ \ \ \ \ \ \ \ \ \ \ \  (b) \ \sum_{n=4}^{\infty}  \dfrac{(-1)^n}{2n^2-5}$$

\
%8
\item (10 pts)
 \emph{Using the definition} find the Taylor series about 2 for the function $f(x)=\dfrac{1}{(x-1)^2}$. You do \emph{not} need to show the Taylor series converges to $f(x)$. \emph{You must use the definition for full credit.}

\


%9
\item (10 pts) Find the Mclaurin series for $f(x)= \frac{1}{2}(e^x+e^{-x})$. You may use any valid method but must explain your reasoning. You do not need to find the radius of convergence or show the series converges to the function.

\

%10
\item (10 pts) We can't use the Fundamental Theorem of Calculus to find $\int_0^1 e^{x^2} \, dx$ because $e^{x^2}$ doesn't have an antiderivative. Instead, express the definite integral as a numerical series. 


\end{enumerate}

\end{document}  