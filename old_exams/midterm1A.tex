\documentclass[12pt, oneside]{article}   	% use "amsart" instead of "article" for AMSLaTeX format
\usepackage{geometry}                		% See geometry.pdf to learn the layout options. There are lots.
\geometry{letterpaper}                   		% ... or a4paper or a5paper or ... 
%\geometry{landscape}                		% Activate for for rotated page geometry
%\usepackage[parfill]{parskip}    		% Activate to begin paragraphs with an empty line rather than an indent
\usepackage{graphicx}				% Use pdf, png, jpg, or eps§ with pdflatex; use eps in DVI mode
\usepackage{amsmath}								% TeX will automatically convert eps --> pdf in pdflatex		
\usepackage{amssymb}

\title{}
\author{MATH1103   \ \ \ \ \ \ Midterm 1 \ \ \ \  September 27, 2021\ \ \ \ Version A\\  \\ Closed book, closed notes; show all your work.}
\date{}							% Activate to display a given date or no date

\begin{document}
\maketitle
%\section{}
%\subsection{}

\leftline{\bf Name: }

\

\begin{enumerate} 

\item 
(10 pts) Find the average value of $f(x)=\sec^2(x)$ on $[0,\pi/4]$.

\

\item (5 pts each) For the following, write True or False, \emph{No justification is needed! Just write True or False.}

\begin{enumerate}

\item  If $f$ is a continuous function on $[a,b]$ and $L_n$ and $R_n$ are the standard left and right hand Riemann sums, then
$$\lim_{n \to \infty} L_n= \lim_{n \to \infty} R_n.$$

\item $$\int_{-2}^2 \left(3x^5+ \dfrac{\sin(x)}{x^2+1} \right) \, dx=0.$$

\item $ \textrm{If } \int_0^8 g(x) \, dx= 12, \textrm{ then } \int_0^2 x^2 g(x^3) \, dx= 6.$

\

\item If $f$ is continuous on $[a,b]$, then $\int_a^b f(x) \, dx$ must be the area between the graph of $f$ and the $x-$axis on $[a,b]$.

\end{enumerate}

\

\item (10 pts) Write the integral that gives the area between the curves $y=x^2$ and $y=2-x$. \emph{ Do not compute the integral, just set it up.}


\newpage

\item (10 pts) Let $R$ be the region bounded by the curves $y=x$, $x+y=8$, and the $y-$axis. Now suppose $R$ is revolved about the line $y=-3$, resulting in the solid of revolution $S$. \emph{Using the washer method}, set up the integral that gives the volume of $S$. \emph{Do not } evaluate the integral; you \emph{do not} need to sketch a picture of the solid, but it's useful to sketch the region $R$.



\

\item (10 pts each) Evaluate the following integrals or derivatives. ($f$ is a continuous function on $(-\infty,\infty)$ in part (d). )

$$ \textrm{(a)} \ \ \int_0^{\sqrt{\pi/2}} x \cos(x^2)\sin(x^2) \, dx \ \ \ \ \ \ \  \ \textrm{(b)} \int \dfrac{a+bx^5}{\sqrt{6ax+bx^6} } \, dx$$

\

$$\textrm{(c)} \int_5^6 x \sqrt{x-5} \, dx \ \ \ \ \ \ \ \ \ \  \textrm{(d)} \ \  \dfrac{d}{dx} \int_{x^2}^{\cos(x)} f(t) \, dt $$


\

\item (10 pts total) On a small planet, the gravitational constant is -12 feet/sec$^2$. Suppose a rock is dropped from an initial height of 150 feet. 

\begin{enumerate}
\item Find its height $s(t)$ after $t$ seconds while it's in motion.

\

\item How long does it take to hit the ground?
\end{enumerate}





\end{enumerate}


\end{document}  