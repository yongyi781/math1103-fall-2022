\documentclass[11pt,oneside]{amsart}
\usepackage{geometry}
\usepackage{amssymb,parskip}
\usepackage[shortlabels]{enumitem}

\theoremstyle{definition}
\newtheorem{problem}{Problem}

\newcommand{\bC}{\mathbb{C}}
\newcommand{\bQ}{\mathbb{Q}}
\newcommand{\bR}{\mathbb{R}}
\newcommand{\bZ}{\mathbb{Z}}

\title{MATH1103 Fall 2022\\
Discussion Sheet 3}

\begin{document}
    \maketitle
    Every discussion you will be assigned three problems and you are encouraged to work in groups.

    \begin{problem}
         Can you recall the statement of Fundamental Theorem of Calculus Part II? What result have you used in the proof(a VERY famous result)? Can you state it? Can you use a graph to show what the result means? Afterwards, can you recal the proof of FCT Part II?
    \end{problem}

    \begin{problem}
       This problem aims to help you distinguish between definite integrals and indefinite integrals.
       \begin{enumerate}
           \item Determine whether the following is definite or indefinite
           $$\int_0^1 x^2 \text{ }dx$$
           \item Determine whether the following is definite or indefinite
           $$\int x \text{ }dx
           $$
           \item In general, what does a definite integral look like and what does an indefinite integral look like? What are the differences can you observe between them, amongst which is the biggest?
       \end{enumerate}
       
       
       
    \end{problem}

    \begin{problem}
       This problem aims to help with the understanding of the method $u$-substitution. In your PS2, you were asked to prove $$\int_{-a}^a f(x) \text{ }dx=0$$ for any odd function $f(x)$. Here's the sketch of the algebraic proof:
           $$
           \begin{aligned}
           \int_{-a}^a f(x) \text{ }dx&=\int_{-a}^ 0 f(x) \text{ }dx+\int_0^a f(x) \text{ }dx\\
           &=-\int_0^{-a} f(x) \text{ }dx+\int_0^a f(x) \text{ }dx\\
           &=-\int_0^a f(-u) (-du)+ \int_0^a f(x) \text{ }dx\text{ (by letting $u=-x$)}\\
           &=-\int_0^a f(u) \text{ }du+\int_0^a f(x) \text{ }dx=0
           \end{aligned}
           $$
        Can you specify the reason why each equality is indeed an equality in the proof above? Especially in the last step, it indicates that $$\int_0^a f(x) \text{ }dx =\int_0^a f(u) \text{ }du$$ Is this always true? Why? Think about why we can do $u$-substitution for integration.
        
    \end{problem}


\end{document}