\documentclass[11pt,oneside]{amsart}
\usepackage{geometry}
\usepackage{amssymb,parskip}
\usepackage[shortlabels]{enumitem}

\theoremstyle{definition}
\newtheorem{problem}{Problem}

\newcommand{\bC}{\mathbb{C}}
\newcommand{\bQ}{\mathbb{Q}}
\newcommand{\bR}{\mathbb{R}}
\newcommand{\bZ}{\mathbb{Z}}

\title{MATH1103 Fall 2022\\
Discussion Sheet 7}

\begin{document}
    \maketitle

    \begin{problem}\textit{Geometric series and decimal expression of a number}. In your HW05, you must have encountered a sum like
    $$
    \sum_{i=1}^\infty (\frac{1}{2})^i=\frac{1}{2}+(\frac{1}{2})^2+(\frac{1}{2})^3+\cdots
    $$
    Did you remember how you have calculated it? Let's examine some general cases.
        \begin{enumerate}
         \item (Finite series). If we define $s_n$ as follows:
         $$
         s_n=\sum_{i=0}^{n-1} q^i=1+q+q^2+\cdots+q^{n-1}, q \not = 1
         $$
         Then there is a formula saying that 
         $$
         s_n=\frac{1-q^n}{1-q},q \not = 1
         $$
         Can you prove it?
         \vfill
         \item Let's examine a geometric series,
         $$
         s=\sum_{i=0}^\infty q^i
         $$
         Can you come up with a formula for $s$? Does this formula hold for any $q \not=1$? If not, what should the domain of $q$ be to make it make sense? How could you prove this formula?
         \vfill
         \item (Decimal expression of a number)
         \newline
         When we think of the number $1/2$, we at the same time knows that it is also $.5$. Similarly, there is an infamous/a famous fact that $1=.99999\cdots$. Have you ever thought about why we have an infinite decimal expression for a number? For instance, if we say a number has the following decimal expression
         $$
         a=.a_1a_2a_3a_4\cdots
         $$
         What does this mean? Why does it make sense?
         \vfill
         \end{enumerate}
    \end{problem}
    
    \begin{problem}(\textit{Epsilon Lemma})
    \newline
    The $\epsilon$ is stated as follows: If the distance between two numbers is smaller than any positive number then the two numbers are equal. 
    \newline
    Mathematically, this means for any number $x,y$, if $|x-y|<\epsilon$ for any positive number $\epsilon$. Then $x=y$.
    \newline
    Can you give some intuitive example to understand this statement? Can you prove it?
    \vfill
    
    \end{problem}
    
    \begin{problem}(Euclid's proof for existence of infinitely many primes)
    \newline
    We know that primes can be listed as $2,3,5,7,11,\cdots$. Indeed, there are infinitely many primes. How can we prove it?..... Huh, interesting!
    \vfill
    \end{problem}

    


\end{document}