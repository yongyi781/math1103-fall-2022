\documentclass[11pt,oneside]{amsart}
\usepackage{geometry}
\usepackage{amssymb,parskip}
\usepackage[shortlabels]{enumitem}

\theoremstyle{definition}
\newtheorem{problem}{Problem}

\newcommand{\bC}{\mathbb{C}}
\newcommand{\bQ}{\mathbb{Q}}
\newcommand{\bR}{\mathbb{R}}
\newcommand{\bZ}{\mathbb{Z}}

\title{MATH1103 Fall 2022\\
Discussion Sheet 11}

\begin{document}
    \maketitle
 
    \begin{problem} In this problem you will uncover many different ways to sum the series
        \[1+2x+3x^2+4x^3+\cdots=\sum_{n=0}^\infty (n+1)x^n.\]
        Spoiler alert: we'll find that the series sums to $1/(1-x)^2$ (whenever it converges).
        \begin{enumerate}[(a)]
            \item In your homework, you were asked to calculate this sum in two different ways. Can you do it again without looking at your homework?
            \vfill
            \item Notice that $1/(1-x)^2$ is the square of $1/(1-x)$. Therefore, it somehow must be true that
            \[(1+x+x^2+\dots)^2=1+2x+3x^2+4x^3+\dots.\]
            See if you can argue why this equation is true by going through the expanding process on the left hand side. For example, I can see that the coefficient of $x$ in $(1+x+x^2+\dots)^2$ should be 2 because when we multiply $(1+x+x^2+\dots)(1+x+x^2\dots)$, the only two ways to get an $x^1$ term are to pick 1 from the first group and $x$ from the second group, or to pick $x$ from the first group and 1 from the second group.
            \vfill
            \item What series do you think equals $1/(1-x)^3$?
            \vfill
        \end{enumerate}
    \end{problem}
    \newpage
    \begin{problem}\textit{Integral Test}.
    The Integral test for convergence of a series states as follows (Let $a_n=f(n)$ for all positive integer $n$, where $f(x)$ is a function):
    $$
    \text{If }\int_1^\infty f(x) \text{ }dx \text{ converges, then }\sum_{n=1}^\infty a_n \text{ converges}.
    $$
    $$
    \text{If }\int_1^\infty f(x) \text{ }dx \text{ diverges, then }\sum_{n=1}^\infty a_n \text{ diverges}.
    $$
    \begin{enumerate}
        \item Recall that the harmonic series
        $$
        \sum_{n=1}^\infty \frac{1}{n}
        $$
        diverges. What kind of technique did we use last time? Indeed, the divergence of harmonic series can also be proved by the integral test!
        \vfill
        \item Consider the series
        $$
        \sum_{n=1}^\infty \frac{1}{n^2}
        $$
        In you HW, you proved that this series converges. What kind of technique did you use? Indeed, the convergence of the series above can also be proved by integral test!
        \vfill
        \item Based on the last two parts, can you generalize the idea of the proof to the general statement as shown at the beginning of the questions?
        \vfill
    \end{enumerate}
    \end{problem}

\end{document}