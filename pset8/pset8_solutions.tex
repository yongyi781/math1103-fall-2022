\documentclass[11pt,oneside]{amsart}
\usepackage{geometry}
\usepackage{amssymb,parskip,mathtools,microtype,pgfplots}
\usepackage[shortlabels]{enumitem}
\usepackage[colorlinks]{hyperref}
\usepackage[most]{tcolorbox}

\usepgfplotslibrary{fillbetween,decorations.softclip}
\pgfplotsset{compat=1.18}

\definecolor{sol}{rgb}{0.1, 0.3, 0.6}

\newtcolorbox{solution}{enhanced, breakable, colframe=sol, title=Solution}

\theoremstyle{definition}
\newtheorem{problem}{Problem}

\theoremstyle{plain}
\newtheorem{theorem}{Theorem}

\newcommand{\bC}{\mathbb{C}}
\newcommand{\bQ}{\mathbb{Q}}
\newcommand{\bR}{\mathbb{R}}
\newcommand{\bZ}{\mathbb{Z}}
\newcommand{\bE}{\mathbb{E}}
\newcommand{\eps}{\varepsilon}
\newcommand{\blank}{\underline{\hspace{1cm}}}
\newcommand{\longblank}{\underline{\hspace{2cm}}}

\DeclareMathOperator{\Var}{Var}

\title{MATH1103 Fall 2022\\
Problem Set 8}

\begin{document}
    \maketitle
    This problem set is due on Wednesday, November 2 at 11:59 pm. Each problem part is worth 3 points. Collaboration is encouraged. In all cases, you must write your own solutions, and and you must cite collaborators and resources used.

    \begin{problem}
        Reeder proves that $r^n\to 0$ for $0<r<1$ using the Binomial theorem and it is understandably hard to understand. In fact we can just use logarithms to prove it, provided we are willing to accept facts about the logarithm from algebra. This problem walks you through the proof.

        \begin{enumerate}[(a)]
            \item For concreteness's sake let's take $r=0.9=9/10$, so we want to prove that $0.9^n\to 0$ as $n\to\infty$. Suppose your opponent hands you $\eps=0.1$. What is the minimum value of $N$ needed so that $0.9^n$ is less than 0.1 away from 0, for all $n\geq N$?
            \begin{solution}
                Note: I forgot to mention in the problem set that $N$ has to be an integer, even though I intended for $N$ to be an integer. I think most or all people fortunately assumed this to be the case.
                
                The condition that $0.9^n$ is less than 0.1 away from 0 for all $n\geq N$ is the condition that $-0.1<0.9^n<0.1$ for all $n\geq N$. Because $0.9^n$ is always positive, this is the same as the condition that $0.9^n<0.1$ for all $n\geq N$. Because $0.9^n$ is decreasing in $n$, the smallest $N$ such that $0.9^N<0.1$ will suffice. The smallest such $N$ is the smallest integer greater than $\log_{0.9}(0.1)$, i.e.\ $\lceil \log_{0.9}(0.1)\rceil$, which when equals 22 using a calculator.
            \end{solution}
            \item Suppose your opponent hands you $\eps=0.01$. What is the minimum value of $N$ needed now?
            \begin{solution}
                By the same logic, we are looking for $\lceil \log_{0.9}(0.01)\rceil$ which is 44.
            \end{solution}
            \item Now beat your opponent even before he hands you any $\eps$. Give a formula/rule that can produce a sufficient value of $N$, as a function of whatever $\eps$ your opponent may give you.
            \begin{solution}
                By the same logic, we are looking for $\lceil \log_{0.9}(\eps)\rceil$. When we are proving that $0.9^n\to 0$ using the $\eps$ definition, we don't actually need to find the smallest possible $N$. Any integer $N$ that is greater than $\log_{0.9}(\eps)$ works too.
            \end{solution}
            \item With this, write your proof in full, in complete sentences. Before you write your proof, write down the answer to the following: What are you proving?
            \begin{solution}
                We are proving that $0.9^n$ converges to 0.

                \textbf{Proof.} Let $\eps>0$ be arbitrary. We pick $N$ to be any integer greater than $\log_{0.9}(\eps)$, so $0.9^N<\eps$. Then for any $n\geq N$ we have
                \[|0.9^n-0|=0.9^n\leq 0.9^N<\eps,\]
                as desired.
            \end{solution}
            \item Generalize your proof to any $0<r<1$.
            \begin{solution}
                Let $\eps>0$ be arbitrary. We pick $N$ to be any integer greater than $\log_{r}(\eps)$, so $r^N<\eps$. Then for any $n\geq N$ we have
                \[|r^n-0|=r^n\leq r^N<\eps,\]
            \end{solution}
        \end{enumerate}
    \end{problem}

    \begin{problem}
        Let $a$ be a real number and let $(x_n)$ be the constant sequence $x_n=a$ for all $n$. Prove that $x_n\to a$, using the $\eps$ definition.
    \end{problem}
    \begin{solution}
        Let $\eps>0$ be arbitrary. We pick $N=1$. For any $n\geq 1$ we have
        \[|x_n-a|=|a-a|=0<\eps,\]
        as desired.

        (As noted in Problem 1(c), you can be cheeky and pick something like $N=42$. This works too. If you did not mention any $N$, your proof is still quasi-understandable but you should make sure you understand why picking something for $N$ is better writing than not mentioning any $N$ at all.)
    \end{solution}

    \begin{problem}
        Prove that $\dfrac{\sin n}n\to 0$ as $n\to\infty$.
        
        \emph{Hint}: You can use the squeeze theorem. Because you use an existing theorem, you won't need to do any $\eps$ reasoning here.
    \end{problem}
    \begin{solution}
        We have the bound $-1\leq \sin n\leq 1$ for all $n$. Therefore, $-\frac 1n\leq \frac{\sin n}n\leq \frac 1n$ for all $n$. Since $-\frac 1n$ and $\frac 1n$ both converge to 0, the squeeze theorem implies that $\frac{\sin n}n$ converges to 0 as well.
    \end{solution}

    \begin{problem}
        Use the Zax Theorem to prove that the sequence $(z_n)$ given by 
        \[z_n=\frac{1}{2}-\frac{1}{3}+\frac{1}{4}-\cdots+\frac{(-1)^n}{n}
        \] 
        converges. (You don't have to find the limit. Note that $z_n$ only makes sense for $n\geq 2$, so we're starting the sequence at $n=2$.) 
    \end{problem}
    \begin{solution}
        By drawing a picture we notice that the even terms seem to always lie above the odd terms. This motivates defining
        \begin{align*}
            x_n &= z_{2n},\quad n\geq 1,\\
            y_n &= z_{2n+1},\quad n\geq 1.
        \end{align*}
        Notice that $z_n$ is the interleaving of $x_n$ and $y_n$, so if we can show that $x_n$ and $y_n$ both converge to the same number, this will show that $z_n$ to the same number, and we win.

        By the Zax theorem, we must show that $x_n$ is decreasing, that $y_n$ is increasing, that $x_n\geq y_n$ for all $n$, and that $x_n-y_n\to 0$ as $n\to\infty$.

        First let's show $x_n$ is decreasing. This follows from the following computation:
        \[\begin{split}
            x_{n+1}-x_n &= z_{2n+2}-z_{2n}\\
            &= \left(\frac 12-\frac13+\dots+\frac{(-1)^{2n+2}}{2n+2}\right)-\left(\frac 12-\frac13+\dots+\frac{(-1)^{2n}}{2n}\right)\\
            &= \frac{(-1)^{2n+1}}{2n+1}+\frac{(-1)^{2n+2}}{2n+2}\\
            &= -\frac 1{2n+1}+\frac 1{2n+2}\\
            &<0.
        \end{split}\]
        In this computation, the third line follows from the fact that all the terms from $\frac 12$ to $\frac{(-1)^{2n}}{2n}$ cancel out, leaving only two terms. The fourth line follows from noting that $-1$ raised to an odd power equals $-1$, and $-1$ raised to an even power equals 1. The last line comes from noticing that $1/(2n+2)$ is smaller than $1/(2n+1)$.

        Now we show that $y_n$ is increasing, this follows from the following computation:
        \[\begin{split}
            y_{n+1}-y_n &= z_{2n+3}-z_{2n+1}\\
            &= \left(\frac 12-\frac13+\dots+\frac{(-1)^{2n+3}}{2n+3}\right)-\left(\frac 12-\frac13+\dots+\frac{(-1)^{2n+1}}{2n+1}\right)\\
            &= \frac{(-1)^{2n+2}}{2n+2}+\frac{(-1)^{2n+3}}{2n+3}\\
            &= \frac 1{2n+2}-\frac 1{2n+3}\\
            &>0.
        \end{split}\]
        
        Next, we show that $x_n\geq y_n$ for all $n$. This follows from the following computation:
        \[\begin{split}
            x_n-y_n &= z_{2n}-z_{2n+1}\\
            &= \left(\frac 12-\frac13+\dots+\frac{(-1)^{2n}}{2n}\right)-\left(\frac 12-\frac13+\dots+\frac{(-1)^{2n+1}}{2n+1}\right)\\
            &= -\frac{(-1)^{2n+1}}{2n+1}\\
            &= \frac 1{2n+1}\\
            &>0.
        \end{split}\]
        (An easy mistake to make is to forget the outer negative sign in the third line, which is there because that term was being subtracted in the second line.)

        Looking at the second to last line of the same computation, we also immediately see that $x_n-y_n$ equals $1/(2n+1)$ and this converges to 0 as $n\to\infty$.
    \end{solution}

    \begin{problem}
        Last week you proved that $0.999\ldots=1$ using the $\eps$-lemma. This time, prove that $0.999\ldots=1$ using geometric series.
    \end{problem}
    \begin{solution}
        The number $0.999\ldots$ can be expressed as
        \[0.9+0.09+0.009+\dots=\frac 9{10}+\frac 9{100}+\frac 9{1000}.\]
        This is a geometric series $x_n=ar^n$ with $a=9/10$ and $r=1/10$. As $|r|<1$, this series converges, and the sum of this series is equal to $a/(1-r)=\frac{9/10}{1-1/10}=\frac{9/10}{9/10}=1$.
    \end{solution}

    \begin{problem}
        Rate the difficulty of each problem (1a, 1b, 1c, 1d, 1e, 2, 3, 4, 5) according to the following scale. Your ratings will collectively let me know which areas are difficult in this class. Thanks for your feedback!
        \begin{itemize}
            \item 1 -- Super easy, barely an inconvenience!
            \item 2 -- Not easy, but I was able to solve the problem on my own by comparing it with an example from class or the textbook.
            \item 3 -- Not easy, but I was able to solve the problem on my own through observations, analysis, and/or creative reasoning.
            \item 4 -- I made some progress but got stuck, and with help, I was able to solve the problem. I feel like I understand it now.
            \item 5 -- I could not start this problem without help, but after getting help I was able to solve the problem. I feel like I understand it now.
            \item 6 -- I could not start this problem without help, but after getting help I was able to solve the problem. However, I still don't feel like I understand what is going on in this problem.
            \item 7 -- I could not solve the problem, even with help.
        \end{itemize}
    \end{problem}
    \begin{solution}
        :)
    \end{solution}
\end{document}
