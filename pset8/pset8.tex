\documentclass[11pt,oneside]{amsart}
\usepackage{geometry}
\usepackage{amssymb,parskip,mathtools}
\usepackage[shortlabels]{enumitem}
\usepackage[colorlinks]{hyperref}

\theoremstyle{definition}
\newtheorem{problem}{Problem}

\theoremstyle{plain}
\newtheorem{theorem}{Theorem}

\newcommand{\bC}{\mathbb{C}}
\newcommand{\bQ}{\mathbb{Q}}
\newcommand{\bR}{\mathbb{R}}
\newcommand{\bZ}{\mathbb{Z}}
\newcommand{\bE}{\mathbb{E}}
\newcommand{\eps}{\varepsilon}
\newcommand{\blank}{\underline{\hspace{1cm}}}
\newcommand{\longblank}{\underline{\hspace{2cm}}}

\DeclareMathOperator{\Var}{Var}

\title{MATH1103 Fall 2022\\
Problem Set 8}

\begin{document}
    \maketitle
    This problem set is due on Wednesday, November 2 at 11:59 pm. Each problem part is worth 3 points. Collaboration is encouraged. In all cases, you must write your own solutions, and and you must cite collaborators and resources used.

    \begin{problem}
        Reeder proves that $r^n\to 0$ for $0<r<1$ using the Binomial theorem and it is understandably hard to understand. In fact we can just use logarithms to prove it, provided we are willing to accept facts about the logarithm from algebra. This problem walks you through the proof.

        \begin{enumerate}[(a)]
            \item For concreteness's sake let's take $r=0.9=9/10$, so we want to prove that $0.9^n\to 0$ as $n\to\infty$. Suppose your opponent hands you $\eps=0.1$. What is the minimum value of $N$ needed so that $0.9^n$ is less than 0.1 away from 0, for all $n\geq N$?
            \item Suppose your opponent hands you $\eps=0.01$. What is the minimum value of $N$ needed now?
            \item Now beat your opponent even before he hands you any $\eps$. Give a formula/rule that can produce a sufficient value of $N$, as a function of whatever $\eps$ your opponent may give you.
            \item With this, write your proof in full, in complete sentences. Before you write your proof, write down the answer to the following: What are you proving?
            \item Generalize your proof to any $0<r<1$.
        \end{enumerate}
    \end{problem}

    \begin{problem}
        Let $a$ be a real number and let $(x_n)$ be the constant sequence $x_n=a$ for all $n$. Prove that $x_n\to a$, using the $\eps$ definition.
    \end{problem}

    \begin{problem}
        Prove that $\dfrac{\sin n}n\to 0$ as $n\to\infty$.
        
        \emph{Hint}: You can use the squeeze theorem. Because you use an existing theorem, you won't need to do any $\eps$ reasoning here.
    \end{problem}

    \begin{problem}
        Use the Zax Theorem to prove that the sequence $(z_n)$ given by 
        \[z_n=\frac{1}{2}-\frac{1}{3}+\frac{1}{4}-\cdots+\frac{(-1)^n}{n}
        \] 
        converges. (You don't have to find the limit. Note that $z_n$ only makes sense for $n\geq 2$, so we're starting the sequence at $n=2$.) 
    \end{problem}

    \begin{problem}
        Last week you proved that $0.999\ldots=1$ using the $\eps$-lemma. This time, prove that $0.999\ldots=1$ using geometric series.
    \end{problem}

    \begin{problem}
        Rate the difficulty of each problem (1a, 1b, 1c, 1d, 1e, 2, 3, 4, 5) according to the following scale. Your ratings will collectively let me know which areas are difficult in this class. Thanks for your feedback!
        \begin{itemize}
            \item 1 -- Super easy, barely an inconvenience!
            \item 2 -- Not easy, but I was able to solve the problem on my own by comparing it with an example from class or the textbook.
            \item 3 -- Not easy, but I was able to solve the problem on my own through observations, analysis, and/or creative reasoning.
            \item 4 -- I made some progress but got stuck, and with help, I was able to solve the problem. I feel like I understand it now.
            \item 5 -- I could not start this problem without help, but after getting help I was able to solve the problem. I feel like I understand it now.
            \item 6 -- I could not start this problem without help, but after getting help I was able to solve the problem. However, I still don't feel like I understand what is going on in this problem.
            \item 7 -- I could not solve the problem, even with help.
        \end{itemize}
    \end{problem}
\end{document}
