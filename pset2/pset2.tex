\documentclass[11pt,oneside]{amsart}
\usepackage{geometry}
\usepackage{amssymb,parskip}
\usepackage[shortlabels]{enumitem}

\theoremstyle{definition}
\newtheorem{problem}{Problem}

\newcommand{\bC}{\mathbb{C}}
\newcommand{\bQ}{\mathbb{Q}}
\newcommand{\bR}{\mathbb{R}}
\newcommand{\bZ}{\mathbb{Z}}

\title{MATH1103 Fall 2022\\
Problem Set 2}

\begin{document}
    \maketitle
    This problem set is due on Wednesday, September 14 at 11:59 pm. Each problem part is worth 3 points. Collaboration is encouraged. In all cases, you must write your own solutions, and and you must cite collaborators and resources used.

    \begin{problem}
        Some exercises.
        \begin{enumerate}[(a)]
            \item Find $\int_a^b 1\,dx$ (in terms of $a$ and $b$).
            \item Find $\int_a^b x^{100}\,dx$.
            \item (Strang 5.7.7) Find
            \[\frac d{dx}\int_x^{x+1}v(t)\,dt.\]
            (The integral here is sometimes called a \emph{running average} of $v$.)
            \item (Strang 5.7.8) Find
            \[\frac d{dx}\left(\frac 1x\int_0^x v(t)\,dt\right).\]
            (The integral here represents the average value of $v$ from 0 to $x$. A hint is to use the product rule for derivatives.)
            \item Let $a,b$ be real numbers. Explain why
            \[\frac {d^2}{dx^2}\int_0^x\int_a^b \sin(u^3)\,du\,dt=0.\]

            \emph{Hint}: This is actually not a deep problem. The lesson here is to not be scared of the ``double'' integral; it's just an integral of an expression which happens to itself be an integral.
        \end{enumerate}
    \end{problem}

    \begin{problem}[Strang 5.4.43]
        If $f(t)$ is an antiderivative of $v(t)$, find an antiderivative of
        \begin{enumerate}[(a)]
            \item $v(t+3)$.
            \item $v(t)+3$.
        \end{enumerate}
    \end{problem}

    \begin{problem}
        Find
        \[\int_0^{76.5}\lfloor x\rfloor\,dx.\]
        Recall that for any real number $x$, the notation $\lfloor x\rfloor$, read \emph{floor} of $x$, means the greatest integer less than or equal to $x$. For example, $\lfloor \pi\rfloor=3$, $\lfloor -\pi\rfloor=-4$, and $\lfloor 15\rfloor=15$.
    \end{problem}
    % \begin{problem}
    %     Recall that the constant acceleration of gravity leads to objects falling in a parabolic manner in real life. If the object is thrown instead, since its horizontal speed is unaffected by gravity (hence constant), we can also say that the trajectory of a thrown object follows a parabola.

    %     Imagine a strange universe where gravity gets stronger linearly over time. What kind of trajectory would a thrown object follow? Justify your description with calculus.
    % \end{problem}

    % \begin{problem}[Strang 5.7.32, modified slightly]
    %     \leavevmode\begin{enumerate}[(a)]
    %         \item You choose a random number uniformly between 0 and 1, and I do the same. What is the probability that your number is greater than twice mine?
    %         \item You choose a random number uniformly between 0 and 1, and I do the same. What is the probability that your number is greater than the square of mine?
    %     \end{enumerate}
    % \end{problem}
    
    \begin{problem}
        Define
        \[f(x)=\begin{cases}1 &\text{$x$ is rational},\\0 &\text{$x$ is irrational}.\end{cases}\]
        It turns out that the definite integral
        \[\int_0^1f(x)\,dx\]
        does not exist! In this problem you will find out why.

        You may take for granted the following fact about the real number line:

        \emph{Between any two distinct real numbers, no matter how close they are to each other, one can find both rational numbers and irrational numbers.}

        Using this fact, show that no matter how large $n$ is, a Riemann sum for the integral $\int_0^1 f(x)\,dx$ with $n$ equal partitions can be made equal to 1 or 0 depending on how you pick the sample points.

        \emph{Remark}: Since the Riemann sums do not approach a limit, the aforementioned definite integral therefore does not exist, and such a function is said to be \emph{non-integrable}.

        \emph{Remark 2}: Of course, the same reasoning applies to any bounds on the integral, not just from 0 to 1. 
    \end{problem}

    \begin{problem}
        Let $f$ be an odd function, meaning that $f(-x)=-f(x)$ for all $x\in\bR$. Give two different proofs that
        \[\int_{-a}^a f(x)\,dx=0\]
        for any real number $a$:
        \begin{enumerate}[(a)]
            \item Graphically using the definition of definite integral as a signed area.
            \item Algebraically using $u$-substitution.
        \end{enumerate}
    \end{problem}

    
    % \item Using substitution, prove that
    % \[\int_0^4 xe^{(x-2)^4}\,dx=2\int_0^4e^{(x-2)^4}\,dx.\]

    % \begin{problem}
    %     Define
    %     \[f(x)=\int_1^x\frac 1t\,dt.\]
    %     Using properties of integrals and $u$-substitution, prove that $f(xy)=f(x)+f(y)$ for all positive real numbers $x$ and $y$.

    %     \emph{Remark}: This property of turning multiplication into addition is one of the defining properties of the logarithm function.
    % \end{problem}

\end{document}
