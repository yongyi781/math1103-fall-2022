\documentclass[11pt,oneside]{amsart}
\usepackage{geometry}
\usepackage{parskip,hyperref}

\pagestyle{empty}

\title{Course information for MATH1103 (Fall 2022)\\
    Calculus II (Math/Science Majors)}

\begin{document}
\maketitle

Calculus has a rich history, interesting mathematics, and many applications. In this class we will
explore two of the main themes: integration and applications in the first half, and sequences and series in the second half. Time permitting, we may explore polar coordinates and parametric equations.

For questions about correct course placement, talk to me or see advice at
\url{https://www.bc.edu/bc-web/schools/mcas/departments/math/undergraduate/about-calculus.html}.

\textbf{Instructor:} Yongyi Chen\\
\textbf{Email:} \href{mailto:yongyi.chen@bc.edu}{\texttt{yongyi.chen@bc.edu}}

\textbf{Teaching assistant:} Yaoying Fu\\
\textbf{Email:} \href{mailto:fugh@bc.edu}{\texttt{fugh@bc.edu}}

\textbf{Lectures:}
\begin{itemize}
    \item Section 1: MWF 1:00 pm--1:50 pm in Gasson Hall 302
    \item Section 2: MWF 2:00 pm--2:50 pm in Gasson Hall 302
\end{itemize}
\textbf{Homework:} Weekly, due on Wednesdays at 11:59 pm.

\textbf{Office:} Maloney 532\\
\textbf{Office hours:}
\begin{itemize}
    \item Yongyi: Fridays 3--4 pm, Wednesdays 3--5 pm in Maloney 532, both in person.
    \item Yaoying: Wednesdays 12--1 pm, Thursadys 12--2 pm in Maloney 537.
\end{itemize}


\section{Course information}
\subsection*{Course website}
On Canvas. There you will find homework assignments, homework solutions, and supplemental course materials.

\subsection*{Course format}
In person. Both office hours are in person. I may change hours or add more office hours based on demand.

\subsection*{Textbooks}
We will be loosely following selected sections from the following two course notes:
\begin{itemize}
    \item Mark Reeder's MATH1103 notes.
    \item Gilbert Strang's calculus textbook, freely available online.
\end{itemize}
Both of these two are linked on Canvas.

As a supplementary resource for light and fun reading, I also suggest reading \emph{Hitchhiker's Guide to Calculus} by Spivak.

\subsection*{Homework}
There will be weekly homework, due on Wednesdays at 11:59 pm. Because homework solutions will be posted on Canvas, late homework will not be accepted. To submit your homework, upload a single PDF file to Gradescope (accessible from within the Canvas assignment page as well).

You are encouraged to collaborate on homework with your classmates, but the work that you turn in must be your own and must be written in your own words.  Working together is good; copying somebody else’s work is plagiarism.

Typesetting your homework using LaTeX is strongly encouraged, but not required.

\subsection*{Discussion}
Yaoying Fu will be running our discussion sections, on Thursdays at 9 am, 10 am, and 11 am. Attendance is strongly encouraged, as you will be able to practice on additional problems, work with classmates, and ask the TA any questions.

\subsection*{Exams and grading}
There will be two in-class exams (50 minutes each) and a final (120 minutes). Final grades will be determined by a weighted average of homework and exam scores.  Homework counts for 20\%, each in-class exam counts for 20\%, and the final counts for 40\%.

All exams will be given in class, in the same classroom as lectures are held in. Dates are as follows:
\begin{itemize}
    \item Midterm 1: Wednesday, October 19
    \item Midterm 2: Wednesday, November 30
    \item Final exam:
    \begin{itemize}
        \item Section 1: Wednesday, December 14 at 12:30 pm
        \item Section 2: Friday, December 16 at 12:30 pm
    \end{itemize}
\end{itemize}
There will be no homework due on the same weeks as exams are held.

\subsection*{Academic integrity}
Cheating of any kind will result in a failing grade for the course and referral to the Dean’s office for disciplinary action.  For more information on academic integrity see \url{https://www.bc.edu/integrity}.

\subsection*{Resources}
Here are some Resources to take advantage of:
\begin{enumerate}
    \item Come to class!
    \item I have office hours, listed above.
    \item The Connors Family Learning Center provides peer tutoring for all Boston College Students. See \url{www.bc.edu/libraries/help/tutoring.html} or call 617-552-0611 to schedule an appoint-
ment, after add/drop.
    \item Math Department Tutoring: This is a drop-in tutoring staffed by math majors.
    \item The Math Department office maintains a list of tutors-for-hire who have indicated their
availability for the term. Contact me if you are interested in being put in touch with a
personal, paid tutor.
\end{enumerate}
If you are a student with a documented disability seeking reasonable accommodations in this
course, please contact the Connors Family Learning Center regarding learning disabilities and
ADHD, or the Disability Services Office (617) 552-3470 regarding all other types of disabilities,
including temporary disabilities. Advance notice and appropriate documentation are required for
accommodations.

\section{List of topics}
\begin{enumerate}
    \item Integration and applications
    \begin{itemize}
        \item Areas and distances
        \item The definite integral
        \item The fundamental theorem of calculus
        \item Integration techniques: substitution, parts, trigonometric integrals, partial fractions, improper integrals
        \item Applications: area, volume, probability
    \end{itemize}
    \item Sequences and series
    \begin{itemize}
        \item Definition of sequences, definition of series, definition of convergence
        \item Convergence theorems: comparison theorem, integral test, alternating series test, ratio test, root test
        \item Power series and Taylor series
    \end{itemize}
    \item Additional topics
    \begin{itemize}
        \item Combining power series and integration
    \end{itemize}
\end{enumerate}

\end{document}