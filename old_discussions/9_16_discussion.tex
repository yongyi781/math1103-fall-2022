\documentclass[11pt, oneside]{article}   	% use "amsart" instead of "article" for AMSLaTeX format
\usepackage{geometry}                		% See geometry.pdf to learn the layout options. There are lots.
\geometry{letterpaper}                   		% ... or a4paper or a5paper or ... 
%\geometry{landscape}                		% Activate for for rotated page geometry
%\usepackage[parfill]{parskip}    		% Activate to begin paragraphs with an empty line rather than an indent
\usepackage{graphicx}				% Use pdf, png, jpg, or eps§ with pdflatex; use eps in DVI mode
\usepackage{amsmath}								% TeX will automatically convert eps --> pdf in pdflatex		
\usepackage{amssymb}

\title{}
\author{Week 3 discussion problems}
\date{}							% Activate to display a given date or no date

\begin{document}
\maketitle
%\section{}
%\subsection{}

\begin{enumerate}

\item

Recall the Fundamental Theorem of Calculus says, if $f$ is continuous on an interval $[a,b]$, then for any $x$ in the interval,
$$\dfrac{d}{dx} \int_a^x \, f(t) \, dt  = f(x).$$ Here you will use properties of the integral and the chain rule to extend this. Calculate the following:


$$  (a) \ \ \dfrac{d}{dx} \int_{e^2}^{3x^2}  \ln(u) \, du \ \ \ \ \ \ \ \ \ (b) \ \ \dfrac{d}{dx} \int_{\cos(x)}^{\pi/2} 3t^2 \, dt \ \ \ \ \ \ 
 \ \ \ \ \ \ (c)  \int_{6^x}^{\ln(x^3)}\left( 5t^2+1 \right) \, dt $$
 
 
\

\item 


Suppose a particle moves along a line and its acceleration (in m/sec$^2$) after $t$ seconds is $a(t)=2t+6$, for $0 \le t \le 60$. Suppose also that its initial velocity is -16 m/sec and its initial position is 5.2 meters.

\begin{enumerate}
\item Find the velocity function $v(t)$.

\

\item Find its position after 5 seconds.

\

\item Find the total distance covered in the first 5 seconds.
\end{enumerate}

\

\item Let's do some physics. Suppose you are standing on a planet with no atmosphere and a constant gravitational constant of $g$ (f/sec$^2$). If a moving particle has acceleration due only to gravity and has velocity $v_0$ at time $0$ and initial height $s_0$, then derive its height function $s(t)$ (which is valid until it hits the ground). If you remember the formula, derive it anyway. On Earth, $g=$ -32 f/sec$^2$. Plug that into the height function; does this look familiar?

\end{enumerate}

\end{document}  