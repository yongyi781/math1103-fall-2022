\documentclass[11pt, oneside]{article}   	% use "amsart" instead of "article" for AMSLaTeX format
\usepackage{geometry}                		% See geometry.pdf to learn the layout options. There are lots.
\geometry{letterpaper}                   		% ... or a4paper or a5paper or ... 
%\geometry{landscape}                		% Activate for for rotated page geometry
%\usepackage[parfill]{parskip}    		% Activate to begin paragraphs with an empty line rather than an indent
\usepackage{graphicx}				% Use pdf, png, jpg, or eps§ with pdflatex; use eps in DVI mode
							% TeX will automatically convert eps --> pdf in pdflatex		
\usepackage{amssymb}

\title{}
\author{Week 1 discussion problems}
\date{}							% Activate to display a given date or no date

\begin{document}
\maketitle
%\section{}
%\subsection{}

\begin{enumerate}
%1
\item Suppose the following table gives the velocity $v(t)$ in meters/sec of an object moving in a straight line after $t$ seconds. Estimate the distance covered in 12 seconds. 

\

\begin{tabular}{|  l | c |  c |  c |  c |  c |  c |  c  |}
\hline
time in seconds & 0 & 2 & 4 & 6 & 8 & 10 & 12 \\
\hline
velocity in m/sec & 12.1 & 15.2 & 20.4 & 36.0 & 25.0 & 17.1 & 15.2 \\
\hline
\end{tabular}

\

(Note that different choices of $x_i^*$ will give different estimates of the distance; just choose one.)

\

%2
\item
Let $f(x)=3x+5$ on the interval [0,4].

\begin{enumerate}
\item Graph $f$.

\item Find $R_4$.

\item Find $R_n$ and simplify the expression. (Hint: it's useful to find $x_0, x_1, x_2, x_3, \ldots$ to find $x_i$.)

\item Using (c), find $\lim_{n \to \infty} R_n$.

\item By our definition, your answer to (d) is $\int_0^4 3x+5 \, dx$ and both of these are the area between the graph and the $x-$axis. Use geometry to find this area and see that these are the same. Finally, pretend we've covered the Fundamental Theorem of Calculus and compute the integral.

\end{enumerate}

%3
\item Let $f(x)=7x^4+2x$ on [2,4]. Compute $\sum_{i=1}^3 f(x_i^*) \Delta x$ where [2,4] is partitioned into 3 equal subintervals and $x_i^*$ is the midpoint of the subinterval $[x_{i-1},x_i]$, for $i=1,2,3.$ Do not simplify your answer, unless you have time to kill.



\end{enumerate}

\end{document}  