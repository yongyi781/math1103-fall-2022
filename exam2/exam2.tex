\documentclass[11pt,oneside]{amsart}
\usepackage[margin=1in]{geometry}
\usepackage{amssymb,parskip,mathtools,microtype}
\usepackage[shortlabels]{enumitem}

\theoremstyle{definition}
\newtheorem{problem}{Problem}

\newcommand{\bC}{\mathbb{C}}
\newcommand{\bF}{\mathbb{F}}
\newcommand{\bQ}{\mathbb{Q}}
\newcommand{\bR}{\mathbb{R}}
\newcommand{\bZ}{\mathbb{Z}}
\newcommand{\bE}{\mathbb{E}}
\newcommand{\eps}{\varepsilon}

\DeclareMathOperator{\Var}{Var}

\title{MATH1103 Fall 2022\\
Exam 2}
\author{Wednesday, November 30, 2022}

\begin{document}
\maketitle

Name: \underline{\hspace{6cm}}

This exam is open notes, but calculators are not allowed. There are 50 points total in this exam. If you do not manage to solve a problem, show a strategy you tried and a reflection on why it did not work, for partial credit.

\begin{problem}
Convergence tests. Any proof will do. You may use results from class and homework.
\begin{enumerate}[(a)]
  \item (5 points) Find, with proof, whether $\displaystyle\sum_{k=0}^\infty \frac{k!}{314159^k}$ converges or diverges.
        \vfill
  \item (5 points) Find, with proof, whether $\displaystyle\sum_{n=314159}^\infty\frac 1{n-1}$ converges or diverges.
        \vfill
\end{enumerate}
\end{problem}

\newpage

\begin{problem}
Power series. You may freely use without proof the power series for $e^x$ we established in class:
\[e^x=\frac 1{0!}+\frac x{1!}+\frac{x^2}{2!}+\frac{x^3}{3!}+\cdots.\]
You do not need to write any series in sigma notation.
\begin{enumerate}[(a)]
  \item (5 points) What is the power series of $e^{-5x}$?
        \vfill
  \item (5 points) The below series converges. (You do not need to prove it.)
        \[\frac 1{0!}-\frac {10}{1!}+\frac {100}{2!}-\frac {1000}{3!}+\cdots.\]
        What is its exact value? This problem does not depend on part (a).
        \vfill
        % \item (5 points) What is the power series of $\displaystyle\int_0^x e^{-t^2}\,dt$?\footnote{This problem is an application of power series to integrate a function that can't be integrated algebraically!}

        %       This question uses part (a). If you could not solve part (a), describe what you would do if you did have a result for part (a).
        %       \vfill
\end{enumerate}
\end{problem}

\newpage

\begin{problem}[10 points]
You are freely given (thanks to Euler) that
\[\frac 1{1^2}+\frac 1{2^2}+\frac 1{3^2}+\cdots=\frac{\pi^2}6.\]
\begin{enumerate}[(a)]
  \item (5 points) Prove, from the above equation, that
        \[\frac 1{2^2}+\frac 1{4^2}+\frac 1{6^2}+\cdots=\frac{\pi^2}{24}.\]
        \vfill
  \item (5 points) Using the result of part (a), find the value of
        \[\frac 1{1^2}+\frac 1{3^2}+\frac 1{5^2}+\cdots.\]
        % If you weren't able to solve part (a), you may use the variable $a$ as a substitute for part (a)'s answer, and the answer you get should be in terms of $a$ rather than $\pi^2/8$.
        \vfill
\end{enumerate}
\end{problem}

\newpage

% \begin{problem}
%   Power series.
%   \begin{enumerate}[(a)]
%     \item (5 points) What function has
%     \[\frac 1{1!}-\frac x{3!}+\frac {x^2}{5!}-\dots=\sum_{n=0}^\infty \frac{(-1)^nx^n}{(2n+1)!}\]
%     as its power series?

%     \emph{Hint}: Square roots may come in useful.
%     \vfill
%     \item (5 points) Deduce from part (a) the value of
%     \[\frac 1{1!}-\frac {2}{3!}+\frac {2^2}{5!}-\cdots=\sum_{n=0}^\infty\frac{(-2)^n}{(2n+1)!}.\]
%     If you did not solve part (a), you can use $f(x)$ to denote what part (a)'s answer is.
%     \vfill
%   \end{enumerate}
% \end{problem}

% Might still be too hard?

\begin{problem}[10 points]
What are the first 10 digits of $\dfrac 1{98}$?

\emph{Note}: A solution using long division will be awarded at most 5 points.

\emph{Hint}: First prove that
\[\frac 1{98}=\frac12 \sum_{n=1}^\infty \frac {2^n}{100^n},\]
then use that result to figure out the digits.
\end{problem}

\newpage

\begin{problem}[10 points]
For $n\geq 1$, let
\[b_n=\begin{cases}
    4^n    & \text{if }n<314159      \\
    271828 & \text{if }n\geq 314159.
  \end{cases}\]
What is $\lim_{n\to\infty}b_n$? Prove it using the $\eps$ definition of limit.
\end{problem}

\end{document}
