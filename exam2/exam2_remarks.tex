\documentclass[11pt,oneside]{amsart}
\usepackage[margin=1in]{geometry}
\usepackage{amssymb,parskip,mathtools,microtype}
\usepackage[shortlabels]{enumitem}

\theoremstyle{definition}
\newtheorem{problem}{Problem}
\newtheorem*{remark}{Remark}

\newcommand{\bC}{\mathbb{C}}
\newcommand{\bF}{\mathbb{F}}
\newcommand{\bQ}{\mathbb{Q}}
\newcommand{\bR}{\mathbb{R}}
\newcommand{\bZ}{\mathbb{Z}}
\newcommand{\bE}{\mathbb{E}}
\newcommand{\eps}{\varepsilon}

\DeclareMathOperator{\Var}{Var}

\title{MATH1103 Fall 2022\\
Exam 2 Remarks}
\author{Wednesday, November 30, 2022}

\begin{document}
\maketitle

Name: \underline{\hspace{6cm}}

This exam is open notes, but calculators are not allowed. There are 50 points total in this exam. If you do not manage to solve a problem, show a strategy you tried and a reflection on why it did not work, for partial credit.

\begin{problem}
Convergence tests. Any proof will do. You may use results from class and homework.
\begin{enumerate}[(a)]
  \item (5 points) Find, with proof, whether $\displaystyle\sum_{k=0}^\infty \frac{k!}{314159^k}$ converges or diverges.
        \begin{remark}
          Using the ratio test, we have
          \[\left|\frac{(k+1)!}{314159^{k+1}}\cdot\frac{314159^k}{k!}\right|=\left|\frac{k+1}{314159}\right|\xrightarrow{k\to\infty}\infty,\]
          from which we conclude the series diverges.
        \end{remark}
        \vfill
  \item (5 points) Find, with proof, whether $\displaystyle\sum_{n=314159}^\infty\frac 1{n-1}$ converges or diverges.
        \begin{remark}
          The series is a tail of the harmonic series, so it diverges.
        \end{remark}
        \vfill
\end{enumerate}
\end{problem}

\newpage

\begin{problem}
Power series. You may freely use without proof the power series for $e^x$ we established in class:
\[e^x=\frac 1{0!}+\frac x{1!}+\frac{x^2}{2!}+\frac{x^3}{3!}+\cdots.\]
You do not need to write any series in sigma notation.
\begin{enumerate}[(a)]
  \item (5 points) What is the power series of $e^{-5x}$?
        \begin{remark}
          You can substitute $-5x$ for $x$ in the power series of $e^x$, or you can calculate $\frac 1{n!}\frac{d^n}{dx^n}\Big|_{x=0}e^{-5x}$ for each $n$. Either way works.
        \end{remark}
        \vfill
  \item (5 points) The below series converges. (You do not need to prove it.)
        \[\frac 1{0!}-\frac {10}{1!}+\frac {100}{2!}-\frac {1000}{3!}+\cdots.\]
        What is its exact value? This problem does not depend on part (a).
        \begin{remark}
          This series equals the power series of $e^x$ at $x=-10$, so the exact value is $e^{-10}$.
        \end{remark}
        \vfill
        % \item (5 points) What is the power series of $\displaystyle\int_0^x e^{-t^2}\,dt$?\footnote{This problem is an application of power series to integrate a function that can't be integrated algebraically!}

        %       This question uses part (a). If you could not solve part (a), describe what you would do if you did have a result for part (a).
        %       \vfill
\end{enumerate}
\end{problem}

\newpage

\begin{problem}[10 points]
You are freely given (thanks to Euler) that
\[\frac 1{1^2}+\frac 1{2^2}+\frac 1{3^2}+\cdots=\frac{\pi^2}6.\]
\begin{enumerate}[(a)]
  \item (5 points) Prove, from the above equation, that
        \[\frac 1{2^2}+\frac 1{4^2}+\frac 1{6^2}+\cdots=\frac{\pi^2}{24}.\]
        \begin{remark}
          Divide the given equation by 4, and use the fact that $2^2=4$.
        \end{remark}
        \vfill
  \item (5 points) Using the result of part (a), find the value of
        \[\frac 1{1^2}+\frac 1{3^2}+\frac 1{5^2}+\cdots.\]
        \begin{remark}
          If you subtract the first equation by the second equation, what remains is $\frac 1{1^2}+\frac 1{3^2}+\frac 1{5^2}+\cdots=\frac{\pi^2}6-\frac{\pi^2}{24}=\frac{\pi^2}8$.
        \end{remark}
        % If you weren't able to solve part (a), you may use the variable $a$ as a substitute for part (a)'s answer, and the answer you get should be in terms of $a$ rather than $\pi^2/8$.
        \vfill
\end{enumerate}
\end{problem}

\newpage

% \begin{problem}
%   Power series.
%   \begin{enumerate}[(a)]
%     \item (5 points) What function has
%     \[\frac 1{1!}-\frac x{3!}+\frac {x^2}{5!}-\dots=\sum_{n=0}^\infty \frac{(-1)^nx^n}{(2n+1)!}\]
%     as its power series?

%     \emph{Hint}: Square roots may come in useful.
%     \vfill
%     \item (5 points) Deduce from part (a) the value of
%     \[\frac 1{1!}-\frac {2}{3!}+\frac {2^2}{5!}-\cdots=\sum_{n=0}^\infty\frac{(-2)^n}{(2n+1)!}.\]
%     If you did not solve part (a), you can use $f(x)$ to denote what part (a)'s answer is.
%     \vfill
%   \end{enumerate}
% \end{problem}

% Might still be too hard?

\begin{problem}[10 points]
What are the first 10 digits of $\dfrac 1{98}$?

\emph{Note}: A solution using long division will be awarded at most 5 points.

\emph{Hint}: First prove that
\[\frac 1{98}=\frac12 \sum_{n=1}^\infty \frac {2^n}{100^n},\]
then use that result to figure out the digits.
\end{problem}
\begin{remark}
  To prove that
  \[\frac 1{98}=\frac12 \sum_{n=1}^\infty \frac {2^n}{100^n},\]
  we notice that the right hand side is a geometric series with common ratio $1/50$ (because $2^n/100^n=(2/100)^n=(1/50)^n$) and with first term $1/2\cdot 1/50=1/100$, so the sum of the geometric series is
  \[\frac{1/100}{1-1/50}=\frac{1/100}{49/50}=\frac 1{100}\cdot \frac{50}{49}=\frac 1{98}.\]
  This proves the hint. Now to use it to figure out the first 10 digits, notice
  \[\begin{split}
    \frac12\sum_{n=1}^\infty\frac{2^n}{100^n} &=\frac12\left( \frac2{10^2}+\frac 4{10^4}+\frac8{10^6}+\frac{16}{10^8}+\frac{32}{10^{10}}+\cdots \right)\\
    &=\frac1{10^2}+\frac 2{10^4}+\frac4{10^6}+\frac{8}{10^8}+\frac{16}{10^{10}}+\cdots \\
    &= 0.0102040816\ldots.
  \end{split}\]
  For fun, here is 1/98 to more digits.
  \[0.01020408163265306122448979591836734693877551020408163265\ldots\]
  Let me know if you notice any other patterns! (For example, does it repeat?)
\end{remark}

\newpage

\begin{problem}[10 points]
For $n\geq 1$, let
\[b_n=\begin{cases}
    4^n    & \text{if }n<314159      \\
    271828 & \text{if }n\geq 314159.
  \end{cases}\]
What is $\lim_{n\to\infty}b_n$? Prove it using the $\eps$ definition of limit.
\end{problem}
\begin{remark}
  The limit is 271828. Proof: Let $\eps>0$ be arbitrarily chosen. Choose $N=314159$. (You could also have chosen $N$ to be any number greater than or equal to 314159 if you wanted.) Then for all $n\geq N$,
  \[|b_n-271828|=|271828-271828|=0<\eps.\]
\end{remark}

\end{document}
