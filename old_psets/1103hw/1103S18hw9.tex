\documentclass[12pt]{article}  

%Run with XeLaTeX
\usepackage
[colorlinks=true, pdfstartview=FitV, linkcolor=blue, citecolor=blue, urlcolor=blue]
{hyperref}

\usepackage{amssymb}  
\usepackage{amsthm}
\usepackage{amsmath}
\usepackage{graphics} 
\usepackage{graphicx} 
%\usepackage[latin1]{inputenc}
\usepackage{tikz}
\usepackage{pgfplots}
\usepackage{wrapfig}
\usepackage{caption}
\usepgfplotslibrary{polar}
\usepackage{ skull }
\usetikzlibrary{decorations.fractals}
%\usepackage{pst-func}

% GNUPLOT required
\usepackage{verbatim}

\linespread{1.3}

%\addtolength{\textwidth}{80pt}
\addtolength{\evensidemargin}{20pt}
\addtolength{\oddsidemargin}{20pt}

%%%%%%%%%%%%%%%%%%%%%%%%%%%%%%%%%%%%%%%%%%%%%%
%  Begin user defined commands

\newcommand{\map}[1]{\xrightarrow{#1}}


\newcommand{\bz}{\mathbb Z}
\newcommand{\bq}{\mathbb Q}
\newcommand{\br}{\mathbb R}
\newcommand{\bc}{\mathbb C}
\newcommand{\al}{\alpha}
\newcommand{\be}{\beta}
\newcommand{\ga}{\gamma}
\newcommand{\de}{\delta}
\newcommand{\ep}{\epsilon}
\DeclareMathOperator{\lub}{l.u.b.}
%  End user defined commands
%%%%%%%%%%%%%%%%%%%%%%%%%%%%%%%%%%%%%%%%%%%%%%


%%%%%%%%%%%%%%%%%%%%%%%%%%%%%%%%%%%%%%%%%%%%%%
% These establish different environments for stating Theorems, Lemmas, Remarks, etc.

\newtheorem{Thm}{Theorem}
\newtheorem{Prop}[Thm]{Proposition}
\newtheorem{Lem}[Thm]{Lemma}
\newtheorem{Cor}[Thm]{Corollary}

\theoremstyle{definition}
\newtheorem{Def}[Thm]{Definition}

\theoremstyle{remark}
\newtheorem{Rem}[Thm]{Remark}
\newtheorem{Ex}[Thm]{Example}

\theoremstyle{definition}
\newtheorem{Exercise}{Problem}

\newenvironment{Solution}{\noindent\textbf{Solution.}}{}

%\renewcommand{\labelenumi}{(\alph{enumi})}
\renewcommand\qedsymbol{QED}


\setlength{\parindent}{0in}
\setlength{\parskip}{8pt}

\DeclareMathOperator{\arcsec}{arcsec}
\DeclareMathOperator{\arccot}{arccot}
\DeclareMathOperator{\arccsc}{arccsc}
\DeclareMathOperator{\LH}{\ \underset{\text{LH}}{=}\ }
\newcommand{\Dep}{\Delta_+}
\newcommand{\Dem}{\Delta_-}
\newcommand{\bu}{\mathbf u}
\newcommand{\bv}{\mathbf v}
\newcommand{\bw}{\mathbf w}

\newcommand{\ora}{\overrightarrow}



\addtolength{\textwidth}{80pt}
\addtolength{\evensidemargin}{-40pt}
\addtolength{\oddsidemargin}{-40pt}
\addtolength{\topmargin}{-80pt}
\addtolength{\textheight}{1.8in}

\setlength{\parindent}{0in}
\setlength{\parskip}{8pt}

\DeclareMathOperator{\arcsinh}{arcsinh}


\begin{document}  

%\author{Your Name}
{\bf MATH 1103 Homework 9 }\\
{\bf Due Friday April 13  2018}

{\bf Practice Problems}

{\bf Practice 1.\ } 

a)\ $\int_0^\pi x\sin x\ dx$,\qquad b)\ $\int_0^b xe^{-x}\ dx$ (where $b$ is a constant)
\qquad c) $\int_1^e \log x\ dx$.
{\bf Practice 2.\ } Let $c$ and $r$ be constants

a)\ Find antiderivatives of 
$e^{cx}, \sin(cx), \cos( cx), (cx)^{r}$. 

b)\ If $F(x)$ is an antiderivative of $f(x)$, find an antiderivative of $f(cx)$. 

{\bf Practice 3.\ } For constants $a,b$ compute the indefinite integrals
\[\int e^{ax}\cos bx\ dx, \qquad \text{and}\qquad \int e^{ax}\sin bx\ dx.\]
[Hint: Integration by parts, and don't give up!]

{\bf Practice 4.\ } Let $f$ be a polynomial.

a) Show that 
\[\int_0^\infty f(x) e^{-x}\ dx=f(0)+\int_0^\infty f'(x) e^{-x}\ dx.\]
(Here, $\int_0^\infty$ is defined to be $\lim\limits_{b\to\infty}\int_0^b$.)

b)\ Find a formula for $\int_0^\infty f(x) e^{-x}\ dx$ in terms of the derivatives $f^{(k)}(0)$. 

c)\ Use your formula to show that $\int_0^\infty x^n e^{-x}\ dx=n!$.



{\bf Practice 5.\ } Let $n$ be a. positive integer. Use integration by parts to show that  
\[\int_0^{\pi/2}\cos^{n+1}(\theta)\ d\theta=\frac{n}{n+1}\int_0^{\pi/2}\cos^{n-1}(\theta)\ d\theta.
\]
and use this to compute $\int_0^{\pi/2}\cos^{n}(\theta)\ d\theta$ for any integer $n$. 
%\begin{Solution} See analogous (in fact the same) solution for $\sin^n\theta$ in section 9.8.1 in the notes. 
%\end{Solution}

{\bf Practice 6.\ } Let $p$ and $q$ be integers, with $p>0$ and $q\geq 0$. 
and define 
\[W_{p,q}=\displaystyle{\int_0^1(1-x^{1/p})^q\ dx.}\]
a)\ Using integration by parts, show that if $q\geq 1$ then 
$\displaystyle{W_{p,q}=\frac{q}{p+q} W_{p,q-1}.}$
\newline
b)\ Using part a), show that 
$\displaystyle{\int_0^1(1-x^{1/p})^q\ dx=\frac{p!\ q!}{(p+q)!}}$

%{ Let 
%\[ u=(1-x^{1/p})^q,\quad dv=dx,\quad\text{so}\quad 
%du=q(1-x^{1/p})^{q-1}\cdot\left(-\frac{x^{-1+1/p}}{p} \right),
%\quad v=x.\]
%Since $uv\big\vert_0^1=0$, we get
%\[\begin{split}
%W_{p,q}&=\int_0^1q(1-x^{1/p})^{q-1}\cdot\left(\frac{x^{-1+1/p}}{p}\right)\cdot x\ dx\\
%&=\frac{q}{p}\int_0^1(1-x^{1/p})^{q-1}\cdot \left(x^{1/p}\right)\ dx\\
%&=\frac{q}{p}\int_0^1(1-x^{1/p})^{q-1}\cdot \left(1+x^{1/p}-1\right)\ dx\\
%&=\frac{q}{p}\int_0^1\left[(1-x^{1/p})^{q-1}-(1-x^{1/p})^{q}\right]\ dx\\
%&=\frac{q}{p}\left(W_{p,q-1}-W_{p,q}\right),
%\end{split}\]
%so
%\[
%\left(1+\frac{q}{p}\right) W_{p,q}=\frac{q}{p}\cdot W_{p,q-1}.
%\]
%Solving for $W_{p,q}$ we get 
%\[W_{p,q}=\frac{q}{p+q}\cdot W_{p,q-1}}\]
%as desired.
%Now we have
%\[W_{p,q}=\frac{q}{(p+q)}W_{p,q-1}=\frac{q}{p+q}\cdot\frac{q-1}{p+q-1}\cdot W_{p,q-2}
%=\cdots
%\frac{q}{p+q}\cdot\frac{q-1}{p+q-1}\cdots\frac{1}{p+1}\cdot W_{p,0}
%=\frac{q!\ p!}{(p+q)!}
%\]


\vskip10pt
Problems to turn in are on the next page
\newpage

{\bf Homework 9 problems to be turned in.}

The goal of this entire assignment is to prove that $\pi$ is an irrational number. In fact you will prove the stronger statement that $\pi^2$ is irrational. Practice problem 4 is a good warmup.


{\bf 1.\ } Suppose $f$ is a polynomial function. Use integration by parts to prove that 
\[\int_0^1 f(x)\sin(\pi x)\ dx=\frac{f(0)+f(1)}{\pi}-\frac{1}{\pi^2}\int_0^1f''(x)\sin(\pi x)\ dx.
\]

{\bf 2.\ } Suppose $f$ is a polynomial  of even degree $2n$.   Use the formula in exercise 1 to prove that 
\[\int_0^1 f(x)\sin(\pi x)\ dx=\sum_{k=0}^{n}(-1)^k\frac{f^{(2k)}(0)+f^{(2k)}(1)}{\pi^{2k+1}}.\]
(Here $f^{(2k)}$ is the $2k$-th derivative of $f$.)



{\bf 3.\ } Fix a positive integer $n$ and let 
$f(x)=\dfrac{x^n(1-x)^n}{n!}$. 
Prove the following. 
\begin{itemize}
\item[a)] 
$f^{(2k)}(0)=f^{(2k)}(1)$. [Hint: How is $f(x)$ related to $f(1-x)$?]
\item[b)]  All derivatives of $f(x)$ take integer values at $x=0$. 
[Hint: How is $f^{(m)}(0)$ related to the coefficient of $x^{m}$ in $f(x)$?]
\item[c)] We have $f(x)\leq \dfrac{1}{n!}$ for all $0\leq x\leq 1$. [Hint: this is just good old differential calculus.]
\end{itemize}

\vskip10pt
\rule{\textwidth}{1pt}

Now assume $\pi^2$ is a rational number, say $\pi^2=\dfrac{a}{b}$ where $a$ and $b$ are integers. Since $\lim_{n\to\infty}\dfrac{a^{n}}{n!}=0$, we can choose a positive integer $n$ so that $\dfrac{a^{n}}{n!}<\frac{1}{2}$. 
With this choice of $n$ let $f(x)$ be the  polynomial 
$f(x)=\dfrac{x^n(1-x)^n}{n!}$ from Problem 3, and let 
\[N= \pi a^{n}\int_0^1 f(x)\sin(\pi x).\]


\rule{\textwidth}{1pt}

\vskip10pt
{\bf 4.\ } Use the results of problems 1, 3a) and 3b) to prove that $N$
is an integer. 


\vskip10pt
{\bf 5.\ } Prove that
$0<N<1$
and explain why this proves $\pi^2$ is irrational. 




\end{document}