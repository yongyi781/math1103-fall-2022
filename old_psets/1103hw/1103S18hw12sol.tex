\documentclass[12pt]{article}  

%Run with XeLaTeX
\usepackage
[colorlinks=true, pdfstartview=FitV, linkcolor=blue, citecolor=blue, urlcolor=blue]
{hyperref}

\usepackage{amssymb}  
\usepackage{amsthm}
\usepackage{amsmath}
\usepackage{graphics} 
\usepackage{graphicx} 
%\usepackage[latin1]{inputenc}
\usepackage{tikz}
\usepackage{pgfplots}
\usepackage{wrapfig}
\usepackage{caption}
\usepgfplotslibrary{polar}
\usepackage{ skull }
\usetikzlibrary{decorations.fractals}
%\usepackage{pst-func}

% GNUPLOT required
\usepackage{verbatim}

\linespread{1.3}

%\addtolength{\textwidth}{80pt}
\addtolength{\evensidemargin}{20pt}
\addtolength{\oddsidemargin}{20pt}

%%%%%%%%%%%%%%%%%%%%%%%%%%%%%%%%%%%%%%%%%%%%%%
%  Begin user defined commands

\newcommand{\map}[1]{\xrightarrow{#1}}


\newcommand{\bz}{\mathbb Z}
\newcommand{\bq}{\mathbb Q}
\newcommand{\br}{\mathbb R}
\newcommand{\bc}{\mathbb C}
\newcommand{\al}{\alpha}
\newcommand{\be}{\beta}
\newcommand{\ga}{\gamma}
\newcommand{\de}{\delta}
\newcommand{\ep}{\epsilon}
\DeclareMathOperator{\lub}{l.u.b.}
%  End user defined commands
%%%%%%%%%%%%%%%%%%%%%%%%%%%%%%%%%%%%%%%%%%%%%%


%%%%%%%%%%%%%%%%%%%%%%%%%%%%%%%%%%%%%%%%%%%%%%
% These establish different environments for stating Theorems, Lemmas, Remarks, etc.

\newtheorem{Thm}{Theorem}
\newtheorem{Prop}[Thm]{Proposition}
\newtheorem{Lem}[Thm]{Lemma}
\newtheorem{Cor}[Thm]{Corollary}

\theoremstyle{definition}
\newtheorem{Def}[Thm]{Definition}

\theoremstyle{remark}
\newtheorem{Rem}[Thm]{Remark}
\newtheorem{Ex}[Thm]{Example}

\theoremstyle{definition}
\newtheorem{Exercise}{Problem}

\newenvironment{Solution}{\noindent\textbf{Solution.}}{}

%\renewcommand{\labelenumi}{(\alph{enumi})}
\renewcommand\qedsymbol{QED}
% End environments 
%%%%%%%%%%%%%%%%%%%%%%%%%%%%%%%%%%%%%%%%%%%%%%%
%Some commands to save paper


\setlength{\parindent}{0in}
\setlength{\parskip}{8pt}

\DeclareMathOperator{\arcsec}{arcsec}
\DeclareMathOperator{\arccot}{arccot}
\DeclareMathOperator{\arccsc}{arccsc}
\DeclareMathOperator{\erf}{erf}
\DeclareMathOperator{\LH}{\ \underset{\text{LH}}{=}\ }
\newcommand{\Dep}{\Delta_+}
\newcommand{\Dem}{\Delta_-}
\newcommand{\bu}{\mathbf u}
\newcommand{\bv}{\mathbf v}
\newcommand{\bw}{\mathbf w}

\newcommand{\ora}{\overrightarrow}



\addtolength{\textwidth}{80pt}
\addtolength{\evensidemargin}{-40pt}
\addtolength{\oddsidemargin}{-40pt}
\addtolength{\topmargin}{-80pt}
\addtolength{\textheight}{1.8in}

\setlength{\parindent}{0in}
\setlength{\parskip}{8pt}

\DeclareMathOperator{\arcsinh}{arcsinh}

%%%%%%%%%%%%%%%%%%%%%%%%%%%%%%%%%%%%%%%%%%%%%%
% Now we're ready to start
%%%%%%%%%%%%%%%%%%%%%%%%%%%%%%%%%%%%%%%%%%%%%%

\begin{document}  

%\author{Your Name}
{\bf MATH 1103 Homework 12 solutions}\\
{\bf Not graded, not to be turned in. Solutions will appear.}




{\bf 1.\ } In class we found the period $T$ of the pendulum of length $\ell$ and initial angular displacement $\al$ to be 
\[T=4\sqrt{\frac{\ell}{g}}\int_0^{\pi/2}\frac{du}{\sqrt{1-r^2\sin^2 u}},\footnote{Note typos in original problem, which had $2\pi$ instead of 4 and $\sin u$ instead of $\sin^2 u$.}
\]
where $r=\sin(\al/2)$. Compute $T$ as an infinite series in $r$. 

\begin{Solution}
The integrand expands as 
\[\frac{1}{\sqrt{1-r^2\sin^2 u}}
=\sum_{k=0}^\infty(-1)^k\binom{-1/2}{k} r^{2k}\sin^{2k}u.
\]
We have 
\[\binom{-1/2}{k}=(-1)^kP_k\qquad \text{and}\qquad
\int_0^{\pi/2}\sin^{2k}u=P_k\cdot\frac{\pi}{2},
\]
so integrating term-by-term we get
\[
T\ =\ 2\pi\sqrt{\frac{\ell}{g}}\sum_{k=0}^\infty P_k^2\cdot  r^{2k}
\ =\ 2\pi\sqrt{\frac{\ell}{g}}\left[
1+\left(\frac{1}{2}\right)^2 r^2
+\left(\frac{1\cdot 3}{2\cdot 4}\right)^2 r^4
+\left(\frac{1\cdot 3\cdot 5}{2\cdot 4\cdot 6}\right)^2 r^6
+\cdots\right].
\]

\end{Solution}



{\bf 2.\ } Suppose $\al=\pi/2$. That is, we let the pendulum go from the ceiling. Compute $T$ in terms of 
$\displaystyle\int_0^{\pi/2}
\frac{d\theta}{\sqrt{\cos \theta}}.
$
(Hint: Use the first integral we found for $T$.)

\begin{Solution} Our first integral for $T$ was 
\[T=4\sqrt{\frac{\ell}{2g}}\int_0^{\al}
\frac{d\theta}{\sqrt{\cos\theta-\cos\al}}=
4\sqrt{\frac{\ell}{2g}}\int_0^{\pi/2}
\frac{d\theta}{\sqrt{\cos\theta}},
\]
since $\al=\pi/2$. 
\end{Solution}

{\bf 3.\ } Show that Wallis' integral $\int_0^1(1-x^{1/p})^q\ dx$ 
has the trigonometric form
\[
\int_0^1(1-x^{1/p})^q\ dx=2p\int_0^{\pi/2}\cos^{2p-1}(u)\sin^{2q+1}(u)\ du
\]
[Hint: Let $x=\cos^{2p}(u)$.]

\begin{Solution} With this substitution we have 
\[(1-x^{1/p})^q=(1-\cos^2(u))^q=\sin^{2q}(u),\qquad
dx=-2p\cos^{2p-1}(u)\cdot\sin(u)\ du,
\]
and the limits $0,1$ change to $\pi/2, 0$, so 
\[
\int_0^1(1-x^{1/p})^q\ dx=
\int_{\pi/2}^0\sin^{2q}(u)\cdot (-2p)\cos^{2p-1}(u)\cdot\sin(u)\ du
=2p\int_0^{\pi/2}\cos^{2p-1}(u)\sin^{2q+1}(u)\ du,
\]
as desired.
\end{Solution}

{\bf 4.\ } Assume that Wallis' guess 
\[
\int_0^1(1-x^{1/p})^q\ dx=\frac{p!\ q!}{(p+q)!}
\]
is correct for any numbers $p,q>-1$. (It is.) Combine the two previous problems to compute the period of the pendulum with $\al=\pi/2$, in terms of factorials. [Hint: select $p$ and $q$ so that the trigonometric integral in 3 becomes the integral in 2. ]

\begin{Solution} To get $\int(\cos \theta)^{-1/2}$, we need $2p-1=-(1/2)$, so $p=1/4$, and $2q+1=0$, so $q=-(1/2)$. Then 3 says 
\[\int_0^{\pi/2}
\frac{d\theta}{\sqrt{\cos\theta}}=
\frac{1}{2p}\int_0^1(1-x^{1/p})^q\ dx=
\frac{1}{2p}\cdot \frac{p!\ q!}{(p+q)!}=
2\cdot\frac{\left(\frac{1}{4}\right)!\ \left(-\frac{1}{2}\right)!}
{\left(-\frac{1}{4}\right)!}=
2\sqrt{\pi}\cdot\frac{\left(\frac{1}{4}\right)!}
{\left(-\frac{1}{4}\right)!}
\]
(the approximate value is $2.662$).
Now from 2 we get
\[
T=4\sqrt{\frac{\ell}{2g}}\int_0^{\pi/2}
\frac{d\theta}{\sqrt{\cos\theta}}=
8\sqrt{\frac{\pi\ell}{2g}}\cdot\frac{\left(\frac{1}{4}\right)!}
{\left(-\frac{1}{4}\right)!}
\]

\end{Solution}

{\bf 5.\ } Let $a,b,c$ be positive constants. The equation 
\[\frac{x^2}{a^2}+\frac{y^2}{b^2}+\frac{z^2}{c^2}=1\]
defines an {\it ellipsoid} in three-dimensional $x,y,z$ space, whose  cross-sections are ellipses.
For each fixed $z$ the slice is an ellipse with equation
\[\frac{x^2}{a_z^2}+\frac{y^2}{b_z^2}=1,\]
where 
\[a_z=a\cdot\sqrt{1-\frac{z^2}{c^2}},\qquad \text{and}\qquad 
b_z=b\cdot\sqrt{1-\frac{z^2}{c^2}}.
\]
Use Cavalieri's principle and a result from your last homework to compute the volume of this ellipsoid. 

\begin{Solution} 
According to Cavalieri, the volume is the integral over $z$ of the area of the slice at $z$. This slice is the ellipse described above. By the result of problem 2 the slice area is 
\[\pi a_z b_z=\pi a b\left(1-\frac{z^2}{c^2}\right),\]
so the volume of the ellipsoid is 
\[
2\int_0^c\pi a b\left(1-\frac{z^2}{c^2}\right)\ dz=
2\pi a b\left(c-\frac{1}{c^2}\cdot \frac{c^3}{3}\right)=
\frac{4\pi a b c}{3}.
\]

\end{Solution} 

{\bf 6.\ } The goal of this problem is to compute the volume of a bagel. We will  apply Cavalieri's principle to the slices made by a bagel slicer. Our bagel is obtained by rotating the circle of radius $a$ centered at $(0,b)$ about the $x$-axis, where $a,b$ are constants with $0<a<b$. And $x$ will vary from $-a$ to $a$. 

a)\ What is the area of the bagel slice at $x$? 

b)\ Integrate the above slices to obtain the volume of the bagel. 
(It is easier to find the volume of half a bagel, then multiply by two.)

\begin{Solution} Each slice is an annulus (the region between two concentric circles)
whose radii $y=b\pm\sqrt{a^2-x^2}$ are the top and bottom halves of the orginal bagel-generating circle.  This slice area is 
\[
\pi\left[(b+\sqrt{a^2-x^2})^2-(b-\sqrt{a^2-x^2})^2\right]=
4\pi b\sqrt{a^2-x^2}.
\]
Therefore the volume of the bagel is 
(making the substitution $x=a\sin\theta$):
\[
2\int_0^a4\pi b\sqrt{a^2-x^2}\ dx=
8\pi a^2 b\int_0^{\pi/2} \cos^2\theta\ d\theta=
8\pi a^2 b\cdot \frac{\pi}{4}=2\pi^2 a^2b.
\]
\end{Solution}

{\bf 7.\ }Find the lengths of the following curves.  

a)\ The graph of $f(x)=\frac{2}{3}(x-1)^{3/2},\ 1\leq x\leq 2$

b)\  The graph of $f(x)=x^{3/2},\ 0\leq x\leq 4$.



\begin{Solution} a)\ The length is $\int_1^2 \sqrt{1+(f')^2}$, and
$f'(x)=(x-1)^{1/2}$, so the length is
\[
\int_1^2\sqrt{1+f'(x)^2} dx=
\int_1^2\sqrt{1+x-1}\ dx=\int_1^2\sqrt{x}\ dx=
\frac{2}{3} x^{3/2}\bigg\vert_1^2=
\frac{2}{3}\left(2^{3/2}-1\right).
\]

b)\ The length is $\int_0^4\sqrt{1+(f')^2}$, and 
$f'(x)=(3/2)x^{1/2}$, so the length is 
\[
\int_0^4\sqrt{1+\frac{9}{4} x}\ dx=
\frac{4}{9}\int_0^9 \sqrt{1+u}\ du=
\frac{8}{27}\cdot \left(10^{3/2}-1\right).
\]

\end{Solution}

\end{document}










 
 
