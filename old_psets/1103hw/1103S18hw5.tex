\documentclass[12pt]{article}  

\usepackage
[colorlinks=true, pdfstartview=FitV, linkcolor=blue, citecolor=blue, urlcolor=blue]
{hyperref}

\usepackage{amssymb}  
\usepackage{amsthm}
\usepackage{amsmath}
\usepackage{graphics} 
\usepackage{graphicx} 
%\usepackage[latin1]{inputenc}
\usepackage{tikz}
\usepackage{pgfplots}
\usepackage{wrapfig}
\usepackage{caption}
\usepgfplotslibrary{polar}
\usepackage{ skull }
\usetikzlibrary{decorations.fractals}


% GNUPLOT required
\usepackage{verbatim}

\linespread{1.3}

%\addtolength{\textwidth}{80pt}
\addtolength{\evensidemargin}{20pt}
\addtolength{\oddsidemargin}{20pt}

%%%%%%%%%%%%%%%%%%%%%%%%%%%%%%%%%%%%%%%%%%%%%%
%  Begin user defined commands

\newcommand{\map}[1]{\xrightarrow{#1}}

\newcommand{\N}{\mathbb N}
\newcommand{\Z}{\mathbb Z}
\newcommand{\Primes}{\mathbb P}
\newcommand{\Q}{\mathbb Q}
\newcommand{\R}{\mathbb R}
\newcommand{\C}{\mathbb C}
\newcommand{\bz}{\mathbb Z}
\newcommand{\bq}{\mathbb Q}
\newcommand{\br}{\mathbb R}
\newcommand{\bc}{\mathbb C}
\newcommand{\al}{\alpha}
\newcommand{\be}{\beta}
\newcommand{\ga}{\gamma}
\newcommand{\de}{\delta}
\newcommand{\ep}{\epsilon}
\DeclareMathOperator{\lub}{l.u.b.}
%  End user defined commands
%%%%%%%%%%%%%%%%%%%%%%%%%%%%%%%%%%%%%%%%%%%%%%


%%%%%%%%%%%%%%%%%%%%%%%%%%%%%%%%%%%%%%%%%%%%%%
% These establish different environments for stating Theorems, Lemmas, Remarks, etc.

\newtheorem{Thm}{Theorem}
\newtheorem{Prop}[Thm]{Proposition}
\newtheorem{Lem}[Thm]{Lemma}
\newtheorem{Cor}[Thm]{Corollary}

\theoremstyle{definition}
\newtheorem{Def}[Thm]{Definition}

\theoremstyle{remark}
\newtheorem{Rem}[Thm]{Remark}
\newtheorem{Ex}[Thm]{Example}

\theoremstyle{definition}
\newtheorem{Exercise}{Problem}

\newenvironment{Solution}{\noindent\textbf{Solution.}}{}

%\renewcommand{\labelenumi}{(\alph{enumi})}
\renewcommand\qedsymbol{QED}
% End environments 
%%%%%%%%%%%%%%%%%%%%%%%%%%%%%%%%%%%%%%%%%%%%%%%
%Some commands to save paper


\setlength{\parindent}{0in}
\setlength{\parskip}{8pt}

\DeclareMathOperator{\arcsec}{arcsec}
\DeclareMathOperator{\arccot}{arccot}
\DeclareMathOperator{\arccsc}{arccsc}
\DeclareMathOperator{\LH}{\ \underset{\text{LH}}{=}\ }
\newcommand{\Dep}{\Delta_+}
\newcommand{\Dem}{\Delta_-}
\newcommand{\bu}{\mathbf u}
\newcommand{\bv}{\mathbf v}
\newcommand{\bw}{\mathbf w}

\newcommand{\ora}{\overrightarrow}



\addtolength{\textwidth}{80pt}
\addtolength{\evensidemargin}{-40pt}
\addtolength{\oddsidemargin}{-40pt}
\addtolength{\topmargin}{-80pt}
\addtolength{\textheight}{1.8in}

\setlength{\parindent}{0in}
\setlength{\parskip}{8pt}

\DeclareMathOperator{\arcsinh}{arcsinh}

%%%%%%%%%%%%%%%%%%%%%%%%%%%%%%%%%%%%%%%%%%%%%%
% Now we're ready to start
%%%%%%%%%%%%%%%%%%%%%%%%%%%%%%%%%%%%%%%%%%%%%%

\begin{document}  

%\author{Your Name}
{\bf MATH 1103 Homework 5}\\
{\bf Due Friday March 2, 2018}

Practice Problems (not to be turned in)

{\bf Practice 1.\ } Find the radius of convergence $R$ of the following power series, and if possible find the sum of the series for $-R<x<R$. 
\[a)\ \sum\limits_{n=0}^\infty \frac{x^n}{2^n}\qquad\qquad
b)\ \sum\limits_{n=0}^\infty \frac{x^n}{n!}\qquad\qquad
c)\ \sum\limits_{n=1}^\infty \frac{n^n}{n!}x^n.
\]

{\bf Practice 2.\ } Suppose there is a number $M$ such that  $|a_n|\leq M$ for all $n$. 

(a)\ Prove that $\sum\limits_{n=0}^\infty a_n x^n$ converges for $|x|<1$.

(b)\ Prove that $\sum\limits_{n=0}^\infty a_n \dfrac{x^n}{n!}$ converges for all $x$. 

{\bf Practice 3.\ } Using only the power series $\exp x=\sum\limits_{n=0}^\infty\dfrac{x^n}{n!}$, prove that 
\begin{enumerate}
\item[(a)] $(\exp x)'=\exp x$
\item[(b)] $\exp (x+y)=(\exp x)(\exp y)$ (challenging - use the binomial expansion). 
\item[(c)] $\exp(-x)=\dfrac{1}{\exp x}$.
\item[(d)] $\exp x>0$ for all $x$. 
\end{enumerate}

{\bf Practice 4.\ }  The functions $\cosh t$ and $\sinh t$ can be defined by the power series
\[\
\cosh t= \sum_{n=0}^\infty \frac{t^{2n}}{(2n)!}\qquad
\sinh t= \sum_{n=0}^\infty \frac{t^{2n+1}}{(2n+1)!}.
\]
Prove that these power series converge for all numbers $t$,  that each is the derivative of the other, and that 
\[\cosh^2(t)-\sinh^2(t)=1.\]
Do all of this working directly with the power series for $\cosh t$ and $\sinh t$; do not use $\exp(t)$. 

 \rule{\textwidth}{1pt}
The homework to be turned in may be found on the next page.


\newpage

{\bf Homework 5 problems to be turned in.}

{\bf 1.\ } a)\ Prove that 
\[
\sum_{n=1}^\infty\frac{x^n}{n}=\log\left(\frac{1}{1-x}\right), 
\]
for $|x|<1$. 
[Hint: differentiate both sides.]

b)\ Use part a) to find the sum of the series
\[\sum_{n=1}^\infty \frac{1}{n2^n}.\]


{\bf 2.\ } Suppose $\sum_{n=0}^\infty a_nx^n$ is a power series with radius of convergence $R$. Assume that all $a_n\neq 0$. Prove that the power series $\sum\limits_{n=1}^\infty na_nx^{n-1}$  has the same radius of convergence $R$.

{\bf 3.\ } Starting with the power series (no trigonometry)
\[\cos x=\sum_{n=0}^\infty (-1)^n\frac{x^{2n}}{(2n)!},\qquad 
\sin x=\sum_{n=0}^\infty (-1)^n\frac{x^{2n+1}}{(2n+1)!}
\]
prove that 

a)\ $(\cos x)'=-\sin x$ and $(\sin x)'=\cos x$.

b)\ $\cos^2x+\sin^2 x=1$ (differentiate both sides). 

c)\ $|\sin x|\leq 1$ and $|\cos x|\leq 1$

d)\ $\lim\limits_{x\to 0}\dfrac{\sin x}{x}=1$ and $\lim\limits_{x\to 0}\dfrac{1-\cos x}{x}=0$. 




{\bf 4.\ } The tangent function $\tan x=\dfrac{\sin x}{\cos x}$ is an odd function, so its power series looks like
\[\tan x=a_1x+a_3x^3+a_5x^5+a_7x^7+\cdots.\]
Find the  coefficients $a_1, a_3, a_5, a_7$ by setting 
\[(\cos x)\cdot(a_1x+a_3x^3+a_5x^5+a_7x^7+\cdots)=\sin x\]
and using the power series for $\cos x$ and $\sin x$. 


{\bf 5.\ } 
This problem concerns analogues of the functions $\cosh t$ and $\sinh t$ (see practice problem 4),
namely the power series 
\[
p(t)= \sum_{n=0}^\infty \frac{t^{3n}}{(3n)!}\qquad
q(t)= \sum_{n=0}^\infty \frac{t^{3n+1}}{(3n+1)!}\qquad
r(t)= \sum_{n=0}^\infty \frac{t^{3n+2}}{(3n+2)!}
\]

a)\ Prove that $p(t)$, $q(t)$ and $r(t)$ converge for all $t$. 

b)\ How are the derivatives of $p(t)$, $q(t)$ and $r(t)$ related to one another? 

c)\ Try  to find  three numbers $x,y,z$ all nonzero, satisfying the equation $x^3+y^3+z^3=3xyz+1$. (Don't be upset if you cannot. This part is not graded!)

d)\ Prove that if  $x=p(t)$, $y=q(t)$, $z=r(t)$ then $x,y,z$ satisfy the equation 
$x^3+y^3+z^3=3xyz+1$. 

[Thus we have infinitely many solutions of the equation $x^3+y^3+z^3=3xyz+1$ given by power series. This equation in $x,y,z$ is somewhat analogous to the hyperbola $x^2-y^2=1$ satisfied by $x=\cosh t, y=\sinh t$. ]





\end{document}









\end{document}

 
 
