\documentclass[12pt]{article}  

\usepackage
[colorlinks=true, pdfstartview=FitV, linkcolor=blue, citecolor=blue, urlcolor=blue]
{hyperref}

\usepackage{amssymb}  
\usepackage{amsthm}
\usepackage{amsmath}
\usepackage{graphics} 
\usepackage{graphicx} 
%\usepackage[latin1]{inputenc}
\usepackage{tikz}
\usepackage{pgfplots}
\usepackage{wrapfig}
\usepackage{caption}
\usepgfplotslibrary{polar}
\usepackage{ skull }
\usetikzlibrary{decorations.fractals}


% GNUPLOT required
\usepackage{verbatim}

\linespread{1.3}

%\addtolength{\textwidth}{80pt}
\addtolength{\evensidemargin}{20pt}
\addtolength{\oddsidemargin}{20pt}

%%%%%%%%%%%%%%%%%%%%%%%%%%%%%%%%%%%%%%%%%%%%%%
%  Begin user defined commands

\newcommand{\map}[1]{\xrightarrow{#1}}

\newcommand{\N}{\mathbb N}
\newcommand{\Z}{\mathbb Z}
\newcommand{\Primes}{\mathbb P}
\newcommand{\Q}{\mathbb Q}
\newcommand{\R}{\mathbb R}
\newcommand{\C}{\mathbb C}
\newcommand{\bz}{\mathbb Z}
\newcommand{\bq}{\mathbb Q}
\newcommand{\br}{\mathbb R}
\newcommand{\bc}{\mathbb C}
\newcommand{\al}{\alpha}
\newcommand{\be}{\beta}
\newcommand{\ga}{\gamma}
\newcommand{\de}{\delta}
\newcommand{\ep}{\epsilon}
\DeclareMathOperator{\lub}{l.u.b.}
%  End user defined commands
%%%%%%%%%%%%%%%%%%%%%%%%%%%%%%%%%%%%%%%%%%%%%%


%%%%%%%%%%%%%%%%%%%%%%%%%%%%%%%%%%%%%%%%%%%%%%
% These establish different environments for stating Theorems, Lemmas, Remarks, etc.

\newtheorem{Thm}{Theorem}
\newtheorem{Prop}[Thm]{Proposition}
\newtheorem{Lem}[Thm]{Lemma}
\newtheorem{Cor}[Thm]{Corollary}

\theoremstyle{definition}
\newtheorem{Def}[Thm]{Definition}

\theoremstyle{remark}
\newtheorem{Rem}[Thm]{Remark}
\newtheorem{Ex}[Thm]{Example}

\theoremstyle{definition}
\newtheorem{Exercise}{Problem}

\newenvironment{Solution}{\noindent\textbf{Solution.}}{}

%\renewcommand{\labelenumi}{(\alph{enumi})}
\renewcommand\qedsymbol{QED}
% End environments 
%%%%%%%%%%%%%%%%%%%%%%%%%%%%%%%%%%%%%%%%%%%%%%%
%Some commands to save paper


\setlength{\parindent}{0in}
\setlength{\parskip}{8pt}

\DeclareMathOperator{\arcsec}{arcsec}
\DeclareMathOperator{\arccot}{arccot}
\DeclareMathOperator{\arccsc}{arccsc}
\DeclareMathOperator{\LH}{\ \underset{\text{LH}}{=}\ }
\newcommand{\Dep}{\Delta_+}
\newcommand{\Dem}{\Delta_-}
\newcommand{\bu}{\mathbf u}
\newcommand{\bv}{\mathbf v}
\newcommand{\bw}{\mathbf w}

\newcommand{\ora}{\overrightarrow}



\addtolength{\textwidth}{80pt}
\addtolength{\evensidemargin}{-40pt}
\addtolength{\oddsidemargin}{-40pt}
\addtolength{\topmargin}{-80pt}
\addtolength{\textheight}{1.8in}

\setlength{\parindent}{0in}
\setlength{\parskip}{8pt}

\DeclareMathOperator{\arcsinh}{arcsinh}

%%%%%%%%%%%%%%%%%%%%%%%%%%%%%%%%%%%%%%%%%%%%%%
% Now we're ready to start
%%%%%%%%%%%%%%%%%%%%%%%%%%%%%%%%%%%%%%%%%%%%%%

\begin{document}  

%\author{Your Name}
{\bf MATH 1103 Homework 5}\\
{\bf Due Friday March 2, 2018}

Practice Problems (not to be turned in)

{\bf Practice 1.\ } Find the radius of convergence $R$ of the following power series, and if possible find the sum of the series for $-R<x<R$. 
\[a)\ \sum\limits_{n=0}^\infty \frac{x^n}{2^n}\qquad\qquad
b)\ \sum\limits_{n=0}^\infty \frac{x^n}{n!}\qquad\qquad
c)\ \sum\limits_{n=1}^\infty \frac{n^n}{n!}x^n.
\]
a)\ $R=2$\quad b)\ $R=\infty$ \quad c)\ $R=1/e$.

{\bf Practice 2.\ } Suppose there is a number $M$ such that  $|a_n|\leq M$ for all $n$. 

(a)\ Prove that $\sum\limits_{n=0}^\infty a_n x^n$ converges for $|x|<1$.

(b)\ Prove that $\sum\limits_{n=0}^\infty a_n \dfrac{x^n}{n!}$ converges for all $x$. 

\begin{Solution}

(a)\ Let $r=|x|<1$. Then $|a_n x^n|\leq Mr^n$ so $\sum |a_nx^n| $ converges by comparison with the geometric series $\sum Mr^n$. Therefore $\sum a_nx^n$ converges as well. 

(b)\ $|a_n \frac{x^n}{n!}|\leq M\frac{|x|^n}{n!}$ and the series $\sum M\frac{|x|^n}{n!}$ converges (see part b) in Practice 1), so $\sum|a_n \frac{x^n}{n!}|$ converges and therefore $\sum a_n \frac{x^n}{n!}$ converges. 

\end{Solution}
{\bf Practice 3.\ } Using only the power series $\exp x=\sum\limits_{n=0}^\infty\dfrac{x^n}{n!}$, prove that 
\begin{enumerate}
\item[(a)] $(\exp x)'=\exp x$
\item[(b)] $\exp (x+y)=(\exp x)(\exp y)$ (challenging - use the binomial expansion). 
\item[(c)] $\exp(-x)=\dfrac{1}{\exp x}$.
\item[(d)] $\exp x>0$ for all $x$. 
\end{enumerate}

\begin{Solution}

(a)\ $(\exp x)'=\sum\limits_{n=1}^\infty n\dfrac{x^{n-1}}{n!}=
\sum\limits_{n=1}^\infty \dfrac{x^{n-1}}{(n-1)!}=
\sum\limits_{n=0}^\infty \dfrac{x^{n}}{n!}=\exp(x).$

(b)\ 
\[\begin{split}\exp(x+y)&=\sum\limits_{n=0}^\infty \dfrac{(x+y)^{n}}{(n)!}=
\sum\limits_{n=0}^\infty \sum\limits_{k=0}^n\dbinom{n}{k}\dfrac{x^ky^{n-k}}{n!}
=\sum\limits_{n=0}^\infty \sum\limits_{p+q=n}\dfrac{x^py^{q}}{p!q!}
=\left(\sum\limits_{p=0}^\infty\dfrac{x^p}{p!}\right)\cdot
\left(\sum\limits_{q=0}^\infty\dfrac{y^q}{q!}\right)\\
&=\exp(x)\cdot\exp(y).
\end{split}
\]

(c)\ $1=\exp(0)=\exp(x-x)=\exp(x)\cdot \exp(-x)$, using (b). 

(d)\ Follows from (c). 



\end{Solution}


{\bf Practice 4.\ }  The functions $\cosh(t)$ and $\sinh (t)$ can be defined by the power series
\[\
\cosh(t)= \sum_{n=0}^\infty \frac{t^{2n}}{(2n)!}\qquad
\sinh(t)= \sum_{n=0}^\infty \frac{t^{2n+1}}{(2n+1)!}.
\]
Prove that these power series converge for all numbers $t$,  that each is the derivative of the other, and that 
\[\cosh^2(t)-\sinh^2(t)=1.\]
Do all of this working directly with the power series; do not use $e^t$. 


\begin{Solution} Let $u=t^2$. $\cosh(t)=p(u)$ where 
\[p(u)=\sum_{n=0}^\infty \frac{u^n}{(2n)!}\]
has all coefficients nonzero. Its radius convergence $R$ is given by the limit
\[\frac{1}{R}=\lim_{n\to\infty}\frac{(2n)!}{(2n+2)!}=\lim_{n\to\infty}\frac{1}{(2n+1)(2n+2)}=0,\]
so $R=\infty$. So $p(u)$ converges for all $u$, hence $\cosh(t)$ converges for all $t$. 
Similarly, the series 
\[q(t)=\sum_{n=0}^\infty \frac{t^{2n}}{(2n+1)!}\]
converges for all $t$, hence so does $tq(t)=\sinh(t)$. 

Easy to check that $(\cosh t)'=\sinh t$ and vice-versa. Differentiating 
\[(\cosh^2(t)-\sinh^2(t))'=2(\cosh t)(\sinh t)-2(\sinh t)(\cosh t)=0,\]
so $\cosh^2(t)-\sinh^2(t)=C$, some constant. Evaluating at $t=0$ gives $C=1$. 
\end{Solution}

 \rule{\textwidth}{1pt}
The homework to be turned in may be found on the next page.


\newpage

{\bf Homework 5 problems to be turned in.}

{\bf 1.\ } a)\ Prove that 
\[
\sum_{n=1}^\infty\frac{x^n}{n}=\log\left(\frac{1}{1-x}\right), 
\]
for $|x|<1$. 
[Hint: differentiate both sides.]

b)\ Use part a) to find the sum of the series
\[\sum_{n=1}^\infty \frac{1}{n2^n}.\]

\begin{Solution}
\[\frac{d}{dx}\sum_{n=1}^\infty\frac{x^n}{n}=\sum_{n=1}^\infty x^{n-1}=
\sum_{n=0}^\infty x^n=\frac{1}{1-x}. 
\]
and $\log\left(\dfrac{1}{1-x}\right)=-\log(1-x)$, so 
\[\frac{d}{dx}\log\left(\frac{1}{1-x}\right)=-\left(\frac{-1}{1-x}\right)=\frac{1}{1-x}.\] 
Therefore 
\[
\sum_{n=1}^\infty\frac{x^n}{n}=\log\left(\frac{1}{1-x}\right)+C 
\]
for some constant $C$. Evaluating at $x=0$ we get $C=0$. 
\end{Solution}

{\bf 2.\ } Suppose $\sum_{n=0}^\infty a_nx^n$ is a power series with radius of convergence $R$. Assume that all $a_n\neq 0$. Prove that the power series $\sum\limits_{n=1}^\infty na_nx^{n-1}$  has the same radius of convergence $R$.
\begin{Solution}
Since all $a_n\neq 0$ we have 
\[R=\lim_{n\to\infty}\frac{|a_n|}{|a_{n+1}|}.\]
Likewise, the series $\sum n a_n x^{n-1}$ has radius of convergence 
\[R'=\lim_{n\to\infty}\frac{|na_n|}{|(n+1)a_{n+1}|}
=R\cdot \lim_{n\to\infty}\frac{n}{n+1}=R,
\]
since  $\lim\limits_{n\to\infty}\dfrac{n}{n+1}=1$. 
\end{Solution}

{\bf 3.\ } Starting with the power series (no trigonometry)
\[\cos x=\sum_{n=0}^\infty (-1)^n\frac{x^{2n}}{(2n)!},\qquad 
\sin x=\sum_{n=0}^\infty (-1)^n\frac{x^{2n+1}}{(2n+1)!}
\]
prove that 

a)\ $(\cos x)'=-\sin x$ and $(\sin x)'=\cos x$.

b)\ $\cos^2x+\sin^2 x=1$ (differentiate both sides). 

c)\ $|\sin x|\leq 1$ and $|\cos x|\leq 1$

d)\ $\lim\limits_{x\to 0}\dfrac{\sin x}{x}=1$ and $\lim\limits_{x\to 0}\dfrac{1-\cos x}{x}=0$. 

\begin{Solution}
a)\ 
\[\begin{split}
(\cos x)'&=\frac{d}{dx}\sum_{n=0}^\infty (-1)^n\frac{x^{2n}}{(2n)!}
=\sum_{n=1}^\infty (-1)^n(2n)\frac{x^{2n-1}}{(2n)!}\\
&=\sum_{n=1}^\infty (-1)^n\frac{x^{2n-1}}{(2n-1)!}
=\sum_{n=0}^\infty (-1)^{n+1}\frac{x^{2n+1}}{(2n+1)!}\\
&=-\sin x. 
\end{split}
\]
The proof that $(\sin x)'=\cos x$ is similar. 

b)\ Differentiating we get
\[2(\cos x)(-\sin x)+2(\sin x)(-\cos x)=0,\]
so 
$\cos^2x+\sin^2 x=C$ for some constant $C$. Evalutating at $x=0$ we find $C=1+0=1$.

c)\ From b) it follows that $\cos^2x\leq 1$ and $\sin^2x\leq 1$. This means $|\cos x|\leq 1$ and $|\sin x|\leq 1$. 

d)\ We have 
\[
\frac{\sin x}{x}=\sum_{n=0}^\infty(-1)^n\frac{x^{2n}}{(2n+1)}=1-\frac{x^2}{3!}+\frac{x^4}{5!}-\cdots\to 1,
\] 
as $x\to 0$. Likewise, 
\[
\frac{1-\cos x}{x}
=\frac{1}{x}\left(1-\sum_{n=0}^\infty(-1)^n\frac{x^{2n}}{(2n)!}\right)
=\frac{1}{x}\left(1-1+\frac{x^2}{2!}-\frac{x^4}{4!}+\cdots\right)
=\frac{x}{2!}-\frac{x^3}{4!}+\cdots\to 0.
\] 
\end{Solution}

{\bf 4.\ } The tangent function $\tan(x)=\dfrac{\sin(x)}{\cos(x)}$ is an odd function, so its power series looks like
\[\tan(x)=a_1x+a_3x^3+a_5x^5+a_7x^7+\cdots.\]
Find the  coefficients $a_1, a_3, a_5, a_7$ by setting 
\[\cos(x)\cdot(a_1x+a_3x^3+a_5x^5+a_7x^7+\cdots)=\sin(x)\]
and using the power series for $\cos(x)$ and $\sin(x)$. 

\begin{Solution}
Setting
\[
\left(1-\frac{x^2}{2!}+\frac{x^4}{4!}-\frac{x^6}{6!}+\cdots\right)\cdot
\left(a_1x+a_3x^3+a_5x^5+a_7x^7+\cdots\right)=
\left(x-\frac{x^3}{3!}+\frac{x^5}{5!}-\frac{x^7}{7!}+\cdots\right),
\]
the left side is 
\[\begin{split}
a_1x+a_3x^3+a_5x^5+a_7x^7+&\cdots\\
-\frac{1}{2!}(a_1x^3+a_3x^5+a_5x^7)+&\cdots\\
+\frac{1}{4!}(a_1x^5+a_3x^7)+&\cdots\\
-\frac{1}{6!}a_1x^7+&\cdots
\end{split}
\]
where $\cdots$ involves terms $x^n$ with $n>7$. 
Comparing coefficients of $x^n$ for $n\leq 7$ we get
\[\begin{split}
a_1&=\ \ 1\\
 a_3-\frac{a_1}{2!}&=-\frac{1}{3!}\\
a_5-\frac{a_3}{2!}+\frac{a_1}{4!}&=\ \ \frac{1}{5!}\\
a_7-\frac{a_5}{2!}+\frac{a_3}{4!}-\frac{a_1}{6!}&=-\frac{x^7}{7!}
\end{split}
\]
and solving we get
\[a_1=1,\quad  a_3=\frac{1}{3},\quad a_5=\frac{2}{15},\quad a_7=\frac{17}{315}.\]
\end{Solution}


{\bf 5.\ } 
This problem concerns analogues of the functions $\cosh(t)$ and $\sinh(t)$ (see practice problem 4),
namely the power series 
\[
p(t)= \sum_{n=0}^\infty \frac{t^{3n}}{(3n)!}\qquad
q(t)= \sum_{n=0}^\infty \frac{t^{3n+1}}{(3n+1)!}\qquad
r(t)= \sum_{n=0}^\infty \frac{t^{3n+2}}{(3n+2)!}
\]

a)\ Prove that $p(t)$, $q(t)$ and $r(t)$ converge for all $t$. 

b)\ How are the derivatives of $p(t)$, $q(t)$ and $r(t)$ related to one another? 

c)\ Try  to find  three numbers $x,y,z$, all nonzero, satisfying the equation $x^3+y^3+z^3=3xyz+1$. (Don't be upset if you cannot. This part is not graded!)

d)\ Prove that if  $x=p(t)$, $y=q(t)$, $z=r(t)$ then $x,y,z$ satisfy the equation 
$x^3+y^3+z^3=3xyz+1$. 

[Thus we have infinitely many solutions of the equation $x^3+y^3+z^3=3xyz+1$ given by power series. This equation in $x,y,z$ is somewhat analogous to the hyperbola $x^2-y^2=1$ satisfied by $x=\cosh(t), y=\sinh(t)$. ]

\begin{Solution}

a)\ We have $p(t)=p_1(t^3)$, where 
\[p_1(t)= \sum_{n=0}^\infty \frac{t^{n}}{(3n)!}\]
has all coefficients nonzero. The series $p_1(t)$ has radius of convergence $R$ where
\[\frac{1}{R}=\lim_{n\to\infty}\frac{(3n)!}{(3n+3)!}=\lim_{n\to\infty}\frac{1}{(3n+1)(3n+2)(3n+3)}=0,\]
So $R=\infty$. Likewise, we have $q(t)=tq_1(t^3)$, and $r(t)=t^2r_1(t^3)$, where 
\[q_1(t)= \sum_{n=0}^\infty \frac{t^{n}}{(3n+1)!}\qquad\text{and}\qquad
r_1(t)= \sum_{n=0}^\infty \frac{t^{n}}{(3n+2)!}
\]
have, respectively, 
\[ \frac{1}{R}=\lim_{n\to\infty}\frac{(3n+1)!}{(3n+4)!}=\lim_{n\to\infty}\frac{1}{(3n+2)(3n+3)(3n+4)}=0,
\]
and
\[ \frac{1}{R}=\lim_{n\to\infty}\frac{(3n+2)!}{(3n+5)!}=\lim_{n\to\infty}\frac{1}{(3n+3)(3n+4)(3n+5)}=0.
\]
Since the series $p_1(t), q_1(t), r_1(t)$ converge for all $t$, so do the series for 
$p(t)=p_1(t^3)$, $ q(t)=tq_1(t^3)$ and $ r(t)=t^2r_1(t^3)$. 

b)\ Since the first term in $p$ is constant, the series for $p'$ starts at $n=1$:
\[p'(t)=\sum_{n=1}^\infty(3n) \frac{t^{3n-1}}{(3n)!}
=\sum_{n=1}^\infty \frac{t^{3n-1}}{(3n-1)!}
=\sum_{n=0}^\infty \frac{t^{3n+2}}{(3n+2)!}=r(t),
\] For $q$ and $r$ the first terms are not constant, so their derived series' start at $n=0$:
\[q'(t)=\sum_{n=0}^\infty(3n+1) \frac{t^{3n}}{(3n+1)!}
=\sum_{n=0}^\infty \frac{t^{3n}}{(3n)!}
=p(t),
\]
\[r'(t)=\sum_{n=0}^\infty(3n+2) \frac{t^{3n+1}}{(3n+2)!}
=\sum_{n=0}^\infty \frac{t^{3n+1}}{(3n+1)!}
=q(t).
\]

c)\ (not graded) One such solution is $x=2/3, y=2/3, z=-1/3$. There are infinitely many rational solutions. 

d)\  We differentiate both sides, using $p'=r, q'=p, r'=q$, part b).
\[(p^3+q^3+r^3)'=3p^2p'+3q^2q'+3r^2r'=3(p^2r+q^2p+r^2q),\]
\[3(pqr)'=3(p'qr+pq'r+pqr')=3(rqr+ppr+pqq),\]
so $p^3+q^3+r^3=3pqr+C$ for some constant $C$. Evaluating at $t=0$, we get $C=1$ since $p(0)=1, q(0)=r(0)=0$. 



\end{Solution}

\end{document}









\end{document}

 
 
