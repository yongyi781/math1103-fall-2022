\documentclass[12pt]{article}  

%Run with XeLaTeX
\usepackage
[colorlinks=true, pdfstartview=FitV, linkcolor=blue, citecolor=blue, urlcolor=blue]
{hyperref}

\usepackage{amssymb}  
\usepackage{amsthm}
\usepackage{amsmath}
\usepackage{graphics} 
\usepackage{graphicx} 
%\usepackage[latin1]{inputenc}
\usepackage{tikz}
\usepackage{pgfplots}
\usepackage{wrapfig}
\usepackage{caption}
\usepgfplotslibrary{polar}
\usepackage{ skull }
\usetikzlibrary{decorations.fractals}
\usepackage{pst-func}

% GNUPLOT required
\usepackage{verbatim}

\linespread{1.3}

%\addtolength{\textwidth}{80pt}
\addtolength{\evensidemargin}{20pt}
\addtolength{\oddsidemargin}{20pt}

%%%%%%%%%%%%%%%%%%%%%%%%%%%%%%%%%%%%%%%%%%%%%%
%  Begin user defined commands

\newcommand{\map}[1]{\xrightarrow{#1}}


\newcommand{\bz}{\mathbb Z}
\newcommand{\bq}{\mathbb Q}
\newcommand{\br}{\mathbb R}
\newcommand{\bc}{\mathbb C}
\newcommand{\al}{\alpha}
\newcommand{\be}{\beta}
\newcommand{\ga}{\gamma}
\newcommand{\de}{\delta}
\newcommand{\ep}{\epsilon}
\DeclareMathOperator{\lub}{l.u.b.}
%  End user defined commands
%%%%%%%%%%%%%%%%%%%%%%%%%%%%%%%%%%%%%%%%%%%%%%


%%%%%%%%%%%%%%%%%%%%%%%%%%%%%%%%%%%%%%%%%%%%%%
% These establish different environments for stating Theorems, Lemmas, Remarks, etc.

\newtheorem{Thm}{Theorem}
\newtheorem{Prop}[Thm]{Proposition}
\newtheorem{Lem}[Thm]{Lemma}
\newtheorem{Cor}[Thm]{Corollary}

\theoremstyle{definition}
\newtheorem{Def}[Thm]{Definition}

\theoremstyle{remark}
\newtheorem{Rem}[Thm]{Remark}
\newtheorem{Ex}[Thm]{Example}

\theoremstyle{definition}
\newtheorem{Exercise}{Problem}

\newenvironment{Solution}{\noindent\textbf{Solution.}}{}

%\renewcommand{\labelenumi}{(\alph{enumi})}
\renewcommand\qedsymbol{QED}
% End environments 
%%%%%%%%%%%%%%%%%%%%%%%%%%%%%%%%%%%%%%%%%%%%%%%
%Some commands to save paper


\setlength{\parindent}{0in}
\setlength{\parskip}{8pt}

\DeclareMathOperator{\arcsec}{arcsec}
\DeclareMathOperator{\arccot}{arccot}
\DeclareMathOperator{\arccsc}{arccsc}
\DeclareMathOperator{\LH}{\ \underset{\text{LH}}{=}\ }
\newcommand{\Dep}{\Delta_+}
\newcommand{\Dem}{\Delta_-}
\newcommand{\bu}{\mathbf u}
\newcommand{\bv}{\mathbf v}
\newcommand{\bw}{\mathbf w}

\newcommand{\ora}{\overrightarrow}



\addtolength{\textwidth}{80pt}
\addtolength{\evensidemargin}{-40pt}
\addtolength{\oddsidemargin}{-40pt}
\addtolength{\topmargin}{-80pt}
\addtolength{\textheight}{1.8in}

\setlength{\parindent}{0in}
\setlength{\parskip}{8pt}

\DeclareMathOperator{\arcsinh}{arcsinh}

%%%%%%%%%%%%%%%%%%%%%%%%%%%%%%%%%%%%%%%%%%%%%%
% Now we're ready to start
%%%%%%%%%%%%%%%%%%%%%%%%%%%%%%%%%%%%%%%%%%%%%%

\begin{document}  

%\author{Your Name}
{\bf MATH 1103 Homework 6}\\
{\bf Due Friday March 16, 2018}

%Practice Problems (not to be turned in)
%{\bf Practice 1.\ } 
%{\bf Practice 2.\ } 
%{\bf Practice 3.\ } 
%{\bf Practice 4.\ }  

 %\rule{\textwidth}{1pt}
 %The homework to be turned in may be found on the next page.

{\bf Homework 6 problems to be turned in.}

{\bf 1.\ } Prove that $\cosh(1)$ is irrational. (Follow the method in section 7.1 for proving $e$ is irrational.)

{\bf 2.\ } Prove that $\cos(1)$ is irrational. (Follow the method in section 7.2 for proving $\sin(1)$ is irrational.)


\rule{\textwidth}{1pt}
The {\bf Bessel functions}\footnote{1784-1846, German astronomer and mathematician}, also called {\bf cylinder functions} are a family of functions 
\[J_0(x),\  J_1(x),\  J_2(x),\ \dots\]
 that give solutions to many physical problems such as planetary motion around the sun, vibrations of circular membranes, and heat distribution in a cylinder. The function $J_n$ is the unique (up  to constant factor) function which is finite at $x=0$ and satisfies {\bf Bessels' differential equation of order $n$}
\[x^2J_n''+xJ_n'+(x^2-n^2)J_n=0.\]


The first two Bessel functions are given by the power series
\[
J_0(x)=\sum_{k=0}^\infty\frac{(-1)^k(x/2)^{2k}}{(k!)^2}\quad\text{and}\quad 
J_1(x)=\sum_{k=0}^\infty\frac{(-1)^k(x/2)^{2k+1}}{k!(k+1)!}.
\]

\psset{xunit=1,yunit=1}

\begin{pspicture}(-8,-.85)(8,1.25)
%\rput(13,0.8){%
 %$\displaystyle J_n(x)=\frac{1}{\pi}\int_0^\pi\cos(x\sin t-nt)\mathrm{d}t$}
\psaxes[Dy=2,Dx=2]{->}(0,0)(-8,-.8)(8,1.2)
\psset{linewidth=1pt}
\psBessel[linecolor=red]{0}{-8}{8}%
\psBessel[linecolor=blue]{1}{-8}{8}%
%\psBessel[linecolor=green]{2}{-28}{28}%
%\psBessel[linecolor=magenta]{3}{-28}{28}%
\end{pspicture}



\rule{\textwidth}{1pt}




{\bf 3.\ } a)\ Identify which graph above is which Bessel function $J_0$ or $J_1$, with justification for your answer. 

b)\ Prove that the power series for $J_0$ and $J_1$ converge for all $x$.

c)\ What is the relation between $J_0'$ and $J_1$\ ?

d)\ Prove that $J_0(x)$ satisfies Bessel's differential equation of order zero. 


{\bf 4.\ }  The purpose of this exercise is to illustrate  Newton's Lemma II in the {\it Principia}, using the function 
\[f(x)=\frac{\sin x}{x}\]
on the interval $[0,\pi]$. 

a) Draw the graph of $f(x)$ and the rectangles illustrating the upper and lower sums when the interval $[0,\pi]$ is divided into four equal parts. 

b) Compute the upper and lower sums, simplifying your answer. What is the difference between these upper and lower sums?

c) Suppose you divided $[0,\pi]$ into $n$ equal parts. Find the difference  between the upper and lower sums without actually computing them individually. How does this relate to what Newton claims in his Lemma II?


%\begin{Solution}
%\end{Solution}

\end{document}









\end{document}

 
 
