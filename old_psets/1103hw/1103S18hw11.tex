\documentclass[12pt]{article}  

%Run with XeLaTeX
\usepackage
[colorlinks=true, pdfstartview=FitV, linkcolor=blue, citecolor=blue, urlcolor=blue]
{hyperref}

\usepackage{amssymb}  
\usepackage{amsthm}
\usepackage{amsmath}
\usepackage{graphics} 
\usepackage{graphicx} 
%\usepackage[latin1]{inputenc}
\usepackage{tikz}
\usepackage{pgfplots}
\usepackage{wrapfig}
\usepackage{caption}
\usepgfplotslibrary{polar}
\usepackage{ skull }
\usetikzlibrary{decorations.fractals}
%\usepackage{pst-func}

% GNUPLOT required
\usepackage{verbatim}

\linespread{1.3}

%\addtolength{\textwidth}{80pt}
\addtolength{\evensidemargin}{20pt}
\addtolength{\oddsidemargin}{20pt}

%%%%%%%%%%%%%%%%%%%%%%%%%%%%%%%%%%%%%%%%%%%%%%
%  Begin user defined commands

\newcommand{\map}[1]{\xrightarrow{#1}}


\newcommand{\bz}{\mathbb Z}
\newcommand{\bq}{\mathbb Q}
\newcommand{\br}{\mathbb R}
\newcommand{\bc}{\mathbb C}
\newcommand{\al}{\alpha}
\newcommand{\be}{\beta}
\newcommand{\ga}{\gamma}
\newcommand{\de}{\delta}
\newcommand{\ep}{\epsilon}
\DeclareMathOperator{\lub}{l.u.b.}
%  End user defined commands
%%%%%%%%%%%%%%%%%%%%%%%%%%%%%%%%%%%%%%%%%%%%%%


%%%%%%%%%%%%%%%%%%%%%%%%%%%%%%%%%%%%%%%%%%%%%%
% These establish different environments for stating Theorems, Lemmas, Remarks, etc.

\newtheorem{Thm}{Theorem}
\newtheorem{Prop}[Thm]{Proposition}
\newtheorem{Lem}[Thm]{Lemma}
\newtheorem{Cor}[Thm]{Corollary}

\theoremstyle{definition}
\newtheorem{Def}[Thm]{Definition}

\theoremstyle{remark}
\newtheorem{Rem}[Thm]{Remark}
\newtheorem{Ex}[Thm]{Example}

\theoremstyle{definition}
\newtheorem{Exercise}{Problem}

\newenvironment{Solution}{\noindent\textbf{Solution.}}{}

%\renewcommand{\labelenumi}{(\alph{enumi})}
\renewcommand\qedsymbol{QED}
% End environments 
%%%%%%%%%%%%%%%%%%%%%%%%%%%%%%%%%%%%%%%%%%%%%%%
%Some commands to save paper


\setlength{\parindent}{0in}
\setlength{\parskip}{8pt}

\DeclareMathOperator{\arcsec}{arcsec}
\DeclareMathOperator{\arccot}{arccot}
\DeclareMathOperator{\arccsc}{arccsc}
\DeclareMathOperator{\erf}{erf}
\DeclareMathOperator{\LH}{\ \underset{\text{LH}}{=}\ }
\newcommand{\Dep}{\Delta_+}
\newcommand{\Dem}{\Delta_-}
\newcommand{\bu}{\mathbf u}
\newcommand{\bv}{\mathbf v}
\newcommand{\bw}{\mathbf w}

\newcommand{\ora}{\overrightarrow}



\addtolength{\textwidth}{80pt}
\addtolength{\evensidemargin}{-40pt}
\addtolength{\oddsidemargin}{-40pt}
\addtolength{\topmargin}{-80pt}
\addtolength{\textheight}{1.8in}

\setlength{\parindent}{0in}
\setlength{\parskip}{8pt}

\DeclareMathOperator{\arcsinh}{arcsinh}

%%%%%%%%%%%%%%%%%%%%%%%%%%%%%%%%%%%%%%%%%%%%%%
% Now we're ready to start
%%%%%%%%%%%%%%%%%%%%%%%%%%%%%%%%%%%%%%%%%%%%%%

\begin{document}  

%\author{Your Name}
{\bf MATH 1103 Homework 11}\\
{\bf Due Friday April 27 2018}



 {\bf Homework 11 problems to be turned in. Practice problems are on the next page.}




{\bf 1.\ } Express the Gaussian integral $G_{n}=\int_0^{\infty} x^n e^{-x^2}\ dx$ in terms of factorials. [Hint: do a $u$-substitution to turn it into a factorial integral.]

%\begin{Solution}
%Let $u=x^2$, so that $du=2x\ dx$ and $x^n\ dx=x^{n-1}\cdot x\ dx=\frac{1}{2}u^{(n-1)/2}\ du.$
%We get
%\[G_n=\tfrac{1}{2}\left(\tfrac{n-1}{2}\right)!\]
%\end{Solution}



{\bf 2.\ } Turn each of the following integrals into a factorial integral and compute it in terms of factorials. ($a$ is a positive constant.)

\[ \text{a)}\ \int_0^\infty e^{-ax^2}\ dx \qquad\qquad
 \text{b)}\ \int_0^\infty e^{-x^3}\ dx\qquad \qquad
 \text{c)}\ \int_0^1\frac{dx}{\sqrt{-\log x}}.
\]

%\begin{Solution} 
%In each case we make a substitution to get the exponential to be $e^{-u}$. 

%a)\ Let $u=ax^2$. Then $x=\sqrt{u/a}$ so $dx=du/(2\sqrt{au})$ and we get
%\[\int_0^\infty e^{-ax^2}\ dx=\frac{1}{2\sqrt{a}}\int_0^\infty u^{-1/2}e^{-u}\ du=
%\frac{1}{2\sqrt{a}}\cdot (-1/2)!=\frac{1}{2}\sqrt{\frac{\pi}{a}}.
%\]

%b)\ Let $u=x^3$. Then $x=u^{1/3}$  so $dx=(1/3)u^{-2/3} du$ and we get
%\[ \int_0^\infty e^{-x^3}\ dx=(1/3)\int_0^\infty u^{-2/3}e^{-u} du=
%(1/3)(-2/3)!=(1/3)!
%\]

%c)\ Let $u=-\log x$. Then $x=e^{-u}$ so $dx=-e^{-u}$,  and the limits $(0,1)$ change to $(\infty,0)$. So we get 
%\[\int_0^1\frac{dx}{\sqrt{-\log x}}=-\int_\infty^0 u^{-1/2} e^{-u}\ du
%=\int_0^\infty u^{-1/2} e^{-u}\ du=(-1/2)!=\sqrt{\pi}.
%\]
%\end{Solution}

{\bf 3.\ } The {\bf error function} $\erf(x)$, is defined by 
\[\erf(x)=\frac{1}{\sqrt{2\pi}}\int_0^x e^{-t^2/2}\ dt.\]
It is used for calculating probabilities. 

\begin{enumerate}
\item[a)] Calculate $\erf(0)$. 
\item[b)] Calculate $\lim\limits_{x\to\infty}\erf(x)$.
\item[c)] Show that $\erf(x)$ is an odd function. 
\item[d)] Determine where $\erf(x)$ is increasing/decreasing, concave up/down. 
\item[e)] Find the power series of $\erf(x)$. 
\end{enumerate}

%\begin{Solution} 
%a)\ 
%\[\erf(0)=\frac{1}{\sqrt{2\pi}}\int_0^0 e^{-x^2/2}\ dx=0.\]
%b)\ 
%\[\lim_{x\to\infty}\erf(x)
%=\frac{1}{\sqrt{2\pi}}\int_0^\infty e^{-x^2/2}\ dx
%=\frac{1}{\sqrt{2\pi}}\cdot\frac{\sqrt\pi}{\sqrt2}=\frac{1}{2}.
%\]
%Here we used the answer to problem 2a), with $a=1/2$. 

%c)\ Make the substitution $u=-t$, so $du=-dt$ and we get
%\[\erf(-x)=\frac{1}{\sqrt{2\pi}}\int_0^{-x} e^{-t^2/2}\ dt
%=
%-\frac{1}{\sqrt{2\pi}}\int_0^{x} e^{-u^2/2}\ dt
%=-\erf(x).
%\]

%d)\ $\erf'(x)=(1/\sqrt{2\pi})e^{-x^2/2}$, which is always positive, so $\erf(x)$ is always increasing. And 
%$\erf''(x)=(1/\sqrt{2\pi})(-x)e^{-x^2/2}$ is positive when $x<0$ and negative for $x>0$.

%e)\ 
%\[\begin{split}
%\erf(x)&=\frac{1}{\sqrt{2\pi}}\int_0^xe^{-t^2/2}\ dt\\
%&=\frac{1}{\sqrt{2\pi}}
%\int_0^x\sum_{k=0}^\infty\frac{(-t^2/2)^k}{k!}\\
%&=\frac{1}{\sqrt{2\pi}}
%\sum_{k=0}^\infty\frac{(-1)^k}{2^kk!}\frac{x^{2k+1}}{2k+1}\\
%&=\frac{1}{\sqrt{2\pi}}\left[
%x-\frac{x^2}{2\cdot 3}+\frac{x^5}{2\cdot 4\cdot 5}-\frac{x^7}{2\cdot 4\cdot 6\cdot 7}+\frac{x^9}{2\cdot 4\cdot 6\cdot 8\cdot 9}-\cdots
%\right]
%\end{split}
%\]
%\end{Solution} 



{\bf 4.\ } Find the area of the interior of the ellipse with equation 
\[\frac{x^2}{a^2}+\frac{y^2}{b^2}=1.\]
(It suffices to compute one-quarter of the area.)

%\begin{Solution} Solving for $y$ we get $y=b\sqrt{1-(x/a)^2}$, so 
%\[\text{Area}=4b\int_0^a\sqrt{1-\frac{x^2}{a^2}}\ dx.\]
%Let $x=a\sin\theta$, so $dx=a\cos\theta$ and the limits change from $[0,a]$ to $[0,\pi/2]$. We get 
%\[\text{Area}
%=4b\int_0^{\pi/2}\sqrt{1-\sin^2\theta}\cdot a\cos\theta\ d\theta
%=4ab\int_0^{\pi/2}\cos^2\theta\ d\theta=
%4ab\cdot P_1\cdot\frac{\pi}{2}=\pi ab.
%\]
%\end{Solution}

{\bf 5.\ } For the circumference of the ellipse in problem 4, some people use the simple formula 
\[
2\pi\sqrt{\frac{a^2+b^2}{2}}=2\pi a\sqrt{1-\frac{E^2}{2}}.
\]

a)\ Find a power series in $E$ for 
\[2\pi a\sqrt{1-\frac{E^2}{2}}\]

b)\ Compare your power series in a) with the power series for the true circumference in equation (9.17) of the notes. Why are they close? Which is bigger?






%%%%%%%%%%%%%%%%%%%%%%%%%%%%%
\newpage
{\bf Practice Problems with solutions (not to be turned in)}

{\bf Practice 1.\ }  Recall the factorial function is given by $s!=\displaystyle \int_0^\infty x^s e^{-x}\ dx$, for $s>-1$. Show that 

a)\ $(s+1)!=(s+1)\cdot s!$. 

b)\ $(-1/2)!=2\int_0^\infty e^{-x^2}\ dx$. 

c)\ $(1/2)!=\int_0^\infty e^{-x^2}\ dx$. 

%\begin{Solution} 
%a)\ $(s+1)!=\int_0^\infty x^{s+1}e^{-x}$. Integrate by parts with $u=x^{s+1}$ and $v'=e^{-x}$. 
%b)\ Using the integral
%\[(-1/2)!=\int_0^\infty x^{-1/2}e^{-x}\ dx\]
%we make the substitution $u=x^{1/2}$ so $du=(1/2)x^{-1/2}\ dx$ and 
%$2du=x^{-1/2}\  dx$, so we get
%\[(-1/2)!=\int_0^\infty 2e^{-u^2}\ dx.\]
%Using a)  with $s=-1/2$ we get 
%\[(1/2)!=(1/2)\cdot(-1/2)!,\]
%so $(-1/2)!=2\cdot(1/2)!$.
%\end{Solution}



\vskip10pt
{\bf Practice 2.\ } 
Show that for any integer $k\geq 0$ we have 
$\left(k-\frac{1}{2}\right)!= k! P_k \sqrt{\pi}$.

\vskip10pt
{\bf Practice 3.\ } Compute the following integrals.

a)\ $\int_0^\infty x^6 e^{-2x}\ dx$.
% $u=2x$, ans. $45/8$

b)\ $\int_0^\infty \sqrt{x} e^{-x^3}$.
%$u=x^3$, ans. $\sqrt{\pi}/8$

c)\ $\int_0^\infty e^{-ax^2}\ dx$, where $a$ is a positive constant.
%$u=\sqrt a\cdot x$, ans. $\frac{1}{2}\sqrt{\pi}{a}$.

d)\ $\int_0^\infty 3^{-4x^2}\ dx$ (use previous problem). 
% use previous problem, ans. $\frac{1}{2}\sqrt{\pi}{4\log 3}$

e)\ $\int_0^\infty e^{-x^a}\ dx$, where $a$ is a positive constant
% u=x^a, ans. $(1/a)!$.

f)\  $\int_0^1 x^8(1-x)^{10}\ dx$.
%ans. $1/(11\cdot 13\cdot 17\cdot 18\cdot 19)$. 

g)\ $\int_0^1\sqrt{\frac{1}{x}-1}\ dx$.
%ans. $\pi/2$

h)\ $\int_0^{\pi/2}\sqrt{\tan\theta}\ d\theta$. 
%ans. $\pi/\sqrt{2}$

i)\ $\int_0^1x^p(-\log x)^q\ dx$, where $p,q$ are postive constants.
%$u=-\log x$, ans. $q!/p^{q+1}$.

j)\ $\int_0^2(4-x^2)^{3/2}\ dx$. 
%$x=2\sin\theta$, ans. $3\pi$.

k)\ $\int_0^2\frac{x^2}{\sqrt{2-x}}\ dx$.
%$u=2x$, ans. $64\sqrt{2}/15$.

l)\ $\int_0^4x^{3/2}(4-x)^{5/2}\ dx$.
%ans. $3\pi/256$

m)\ $\int_0^3\frac{1}{\sqrt{3x-x^2}}\ dx$.
%ans. $\pi$

n)\ $\int_0^\infty x^a e^{-x^2}\ dx$, where $a$ is a positive constant.
%ans. $(1/2)(\frac{n-1}{2})!$

o)\ Find a recursion formula for the integrals 
\[J_n=\int_0^1 \frac{x^n}{\sqrt{1-x^4}}\ dx.\]
[Hint: Integrate by parts, with $u=x^{n-3}(1-x^4)$.]
%ans. $J_n=\frac{n-3}{n-1} J_{n-4}$. 
%16.\ $\int_0^1\sqrt{1-x^4}\ dx$.

%17.\ $\int_0^\infty\frac{x^{p-1}}{1+x}\ dx$, where $p$ is a constant, $0<p<1$.

p)\ $\int_0^\infty\frac{1}{1+x^2}\ dx$
%ans. $\pi/4$

q)\ $\int_0^\infty \frac{1}{(1+x)^2}\ dx$
%ans. $1$


\end{document}










 
 
