\documentclass[12pt]{article}  

%Run with XeLaTeX
\usepackage
[colorlinks=true, pdfstartview=FitV, linkcolor=blue, citecolor=blue, urlcolor=blue]
{hyperref}

\usepackage{amssymb}  
\usepackage{amsthm}
\usepackage{amsmath}
\usepackage{graphics} 
\usepackage{graphicx} 
%\usepackage[latin1]{inputenc}
\usepackage{tikz}
\usepackage{pgfplots}
\usepackage{wrapfig}
\usepackage{caption}
\usepgfplotslibrary{polar}
\usepackage{ skull }
\usetikzlibrary{decorations.fractals}
\usepackage{pst-func}

% GNUPLOT required
\usepackage{verbatim}

\linespread{1.3}

%\addtolength{\textwidth}{80pt}
\addtolength{\evensidemargin}{20pt}
\addtolength{\oddsidemargin}{20pt}

%%%%%%%%%%%%%%%%%%%%%%%%%%%%%%%%%%%%%%%%%%%%%%
%  Begin user defined commands

\newcommand{\map}[1]{\xrightarrow{#1}}


\newcommand{\bz}{\mathbb Z}
\newcommand{\bq}{\mathbb Q}
\newcommand{\br}{\mathbb R}
\newcommand{\bc}{\mathbb C}
\newcommand{\al}{\alpha}
\newcommand{\be}{\beta}
\newcommand{\ga}{\gamma}
\newcommand{\de}{\delta}
\newcommand{\ep}{\epsilon}
\DeclareMathOperator{\lub}{l.u.b.}
%  End user defined commands
%%%%%%%%%%%%%%%%%%%%%%%%%%%%%%%%%%%%%%%%%%%%%%


%%%%%%%%%%%%%%%%%%%%%%%%%%%%%%%%%%%%%%%%%%%%%%
% These establish different environments for stating Theorems, Lemmas, Remarks, etc.

\newtheorem{Thm}{Theorem}
\newtheorem{Prop}[Thm]{Proposition}
\newtheorem{Lem}[Thm]{Lemma}
\newtheorem{Cor}[Thm]{Corollary}

\theoremstyle{definition}
\newtheorem{Def}[Thm]{Definition}

\theoremstyle{remark}
\newtheorem{Rem}[Thm]{Remark}
\newtheorem{Ex}[Thm]{Example}

\theoremstyle{definition}
\newtheorem{Exercise}{Problem}

\newenvironment{Solution}{\noindent\textbf{Solution.}}{}

%\renewcommand{\labelenumi}{(\alph{enumi})}
\renewcommand\qedsymbol{QED}
% End environments 
%%%%%%%%%%%%%%%%%%%%%%%%%%%%%%%%%%%%%%%%%%%%%%%
%Some commands to save paper


\setlength{\parindent}{0in}
\setlength{\parskip}{8pt}

\DeclareMathOperator{\arcsec}{arcsec}
\DeclareMathOperator{\arccot}{arccot}
\DeclareMathOperator{\arccsc}{arccsc}
\DeclareMathOperator{\LH}{\ \underset{\text{LH}}{=}\ }
\newcommand{\Dep}{\Delta_+}
\newcommand{\Dem}{\Delta_-}
\newcommand{\bu}{\mathbf u}
\newcommand{\bv}{\mathbf v}
\newcommand{\bw}{\mathbf w}

\newcommand{\ora}{\overrightarrow}



\addtolength{\textwidth}{80pt}
\addtolength{\evensidemargin}{-40pt}
\addtolength{\oddsidemargin}{-40pt}
\addtolength{\topmargin}{-80pt}
\addtolength{\textheight}{1.8in}

\setlength{\parindent}{0in}
\setlength{\parskip}{8pt}

\DeclareMathOperator{\arcsinh}{arcsinh}

%%%%%%%%%%%%%%%%%%%%%%%%%%%%%%%%%%%%%%%%%%%%%%
% Now we're ready to start
%%%%%%%%%%%%%%%%%%%%%%%%%%%%%%%%%%%%%%%%%%%%%%

\begin{document}  

%\author{Your Name}
{\bf MATH 1103 Homework 6 Solutions}\\
{\bf Due Friday March 16, 2018}

%Practice Problems (not to be turned in)
%{\bf Practice 1.\ } 
%{\bf Practice 2.\ } 
%{\bf Practice 3.\ } 
%{\bf Practice 4.\ }  

 %\rule{\textwidth}{1pt}
 %The homework to be turned in may be found on the next page.

{\bf Homework 6 problems to be turned in.}

{\bf 1.\ } Prove that $\cosh(1)$ is irrational. (Follow the method in section 7.1 for proving $e$ is irrational.)

\begin{Solution}We know that 
\[\cosh 1=\sum_{k=0}^\infty \frac{1}{(2k)!}=1+\frac{1}{2!}+\frac{1}{4!}+\cdots.\]
Suppose $\cosh 1=a/b$ where $a,b$ are positive integers. 
We will find an integer $N$ such that $0<N<1$. This contradiction will prove that $\cosh 1$ is irrational. 

Choose an integer $n$ such that $2n>b$. Then $(2n)!\cdot \cosh 1=(2n)!\cdot \dfrac{a}{b}$ is an integer. Next consider the partial sum
\[S_n=\sum_{k=0}^{n}\dfrac{1}{(2k)!}.\]
 Each denominator $(2k)!$ divides $(2n)!$, so $(2n)!\cdot ch_n$ is an integer. Therefore the number 
\[N=(2n)!\cdot (\cosh 1-S_n)\]
is also an integer. On the other hand,
\[\begin{split}
\cosh1-S_n
&=\cosh1-\left(1+\frac{1}{2!}+\frac{1}{4!}+\cdots+\frac{1}{(2n)!}\right)\\
&\\
&=\frac{1}{(2n+2)!}+\frac{1}{(2n+4)!}+\frac{1}{(2n+6)!}+\cdots\\
&\\
&=\frac{1}{(2n)!}\left[\frac{1}{(2n+1)(2n+2)}+\frac{1}{(2n+1)(2n+2)(2n+3)(2n+4)}+\cdots\right].
\end{split}
\]
To simplify the notation, let $m=(2n+1)(2n+2)$. Then 
\[\cosh1-S_n\leq 
\frac{1}{(2n)!}\left[\frac{1}{m}+\frac{1}{m^2}+\cdots\right]=\frac{1}{(2n)!}\cdot \frac{1}{m-1}<1,
\]
so $0<N<1$, a contradiction. Therefore $\cosh 1$ is irrational. 



\end{Solution}

\newpage
{\bf 2.\ } Prove that $\cos(1)$ is irrational. (Follow the method in section 7.2 for proving $\sin(1)$ is irrational.)

\begin{Solution} We know that 
\[\cos 1=\sum_{k=0}^\infty \frac{(-1)^k}{(2k)!}=1-\frac{1}{2!}+\frac{1}{4!}-\cdots.\]
Suppose $\cos(1)=a/b$ where $a,b$ are positive integers. 
We will find an integer $N$ such that $0<N<1$. This contradiction will prove that $\cos 1$ is irrational. 

 
Choose an integer $n$ such that $2n>b$. Then $(2n)!\cdot \cos 1=(2n)!\cdot \dfrac{a}{b}$ is an integer. Next consider the partial sum
\[c_n=\sum_{k=0}^{n-1}\dfrac{(-1)^k}{(2k)!}.\]
 Each denominator $(2k)!$ divides $(2n)!$, so $(2n)!\cdot c_n$ is an integer. Therefore the number 
\[N=(2n)!\cdot (\cos 1-c_n)\]
is also an integer. On the other hand,
\[\begin{split}
|\cos1-c_n|
&=\left |\cos1-\left(1-\frac{1}{2!}+\frac{1}{4!}+\cdots+(-1)^{n-1}\frac{1}{(2n-2)!}\right)\right|\\
&\\
&=\frac{1}{(2n)!}-\frac{1}{(2n+2)!}+\frac{1}{(2n+4)!}-\cdots\\
&\\
&=\frac{1}{(2n)!}\left[1-\frac{1}{(2n+1)(2n+2)}+\frac{1}{(2n+1)(2n+2)(2n+3)(2n+4)}-\cdots\right]\\
\end{split}
\]
so 
\[N=(2n)!\cdot |\cos1-c_n|=1-\frac{1}{(2n+1)(2n+2)}+\frac{1}{(2n+1)(2n+2)(2n+3)(2n+4)}-\cdots\]
This is an alternating series, in which the even and odd partial sums form a Zax pair of sequences converging to $N$ in the middle. Thus, 
\[0<1-\frac{1}{(2n+1)(2n+2)}< N<1.\]
Therefore $0<N<1$. This contradiction proves that  $\cos 1$ is irrational.\qed


\end{Solution}

\newpage
\rule{\textwidth}{1pt}
The {\bf Bessel functions}\footnote{Friedrich Bessel, 1784-1846, German astronomer and mathematician}, also called {\bf cylinder functions} are a family of functions 
\[J_0(x),\  J_1(x),\  J_2(x),\ \dots\]
 that give solutions to many physical problems such as planetary motion around the sun, vibrations of circular membranes, and heat distribution in a cylinder. The function $J_n$ is the unique (up  to constant factor) function which is finite at $x=0$ and satisfies {\bf Bessels' differential equation of order $n$}
\[x^2J_n''+xJ_n'+(x^2-n^2)J_n=0.\]


The first two Bessel functions are given by the power series
\[
J_0(x)=\sum_{k=0}^\infty\frac{(-1)^k(x/2)^{2k}}{(k!)^2}\quad\text{and}\quad 
J_1(x)=\sum_{k=0}^\infty\frac{(-1)^k(x/2)^{2k+1}}{k!(k+1)!}.
\]

\psset{xunit=1,yunit=1}

\begin{pspicture}(-8,-.85)(8,1.25)
%\rput(13,0.8){%
 %$\displaystyle J_n(x)=\frac{1}{\pi}\int_0^\pi\cos(x\sin t-nt)\mathrm{d}t$}
\psaxes[Dy=2,Dx=2]{->}(0,0)(-8,-.8)(8,1.5)
\psset{linewidth=1pt}
\psBessel[linecolor=red]{0}{-8}{8}%
\psBessel[linecolor=blue]{1}{-8}{8}%
%\psBessel[linecolor=green]{2}{-28}{28}%
%\psBessel[linecolor=magenta]{3}{-28}{28}%
\end{pspicture}



\rule{\textwidth}{1pt}




{\bf 3.\ } a)\ Identify which graph above is which Bessel function $J_0$ or $J_1$, with justification for your answer. 

b)\ Prove that the power series for $J_0$ and $J_1$ converge for all $x$.

c)\ What is the relation between $J_0'$ and $J_1$\ ?

d)\ Prove that $J_0(x)$ satisfies Bessel's differential equation of order zero. 

\begin{Solution}

a)\  From the power series $J_0(0)=1$ and $J_1(0)=0$, which distinguishes the graphs. 

b)\ Let $t=(-x/2)^2$. Then $J_0(x)=\sum\limits_{k=0}^\infty\dfrac{t^{k}}{(k!)^2}$
has coefficients 
\[a_k=\frac{1}{(k!)^2}>0.\]
Its radius $R$ of convergence is given by
\[\frac{1}{R}=\lim_{k\to\infty}\frac{(k!)^2}{((k+1)!)^2}=\lim_{k\to\infty}\frac{1}{(k+1)^2}=0,\]
so $R=\infty$.  This means $J_0$ converges for all $t$, hence for all $x$. 

Also $J_1(x)=\dfrac{x}{2}\sum\limits_{k=0}^\infty\dfrac{t^{k}}{k!(k+1)!}$ where the infinite series 
has coefficients 
\[a_k=\frac{1}{k!(k+1)!}>0,\]
and its $R$ is given by 
\[\frac{1}{R}=\lim_{k\to\infty}\frac{k!(k+1)!}{(k+1)!(k+2)!}=\lim_{k\to\infty}\frac{1}{(k+1)(k+2)}=0,\]
so $R=\infty$ for $J_1$ as well. 

c)\ Differentiating term-by-term, we have 
\[J_0'(x)=\sum_{k=1}^\infty\frac{(-1)^kk(x/2)^{2k-1}}{(k!)^2}
=\sum_{k=1}^\infty\frac{(-1)^k(x/2)^{2k-1}}{k!(k-1)!}
=\sum_{k=0}^\infty\frac{(-1)^{k+1}(x/2)^{2k+1}}{k!(k+1)!}
=-J_1(x).
\]

d)\ We have to show that $xJ_0''+J_0'+xJ_0=0$. We have 
\[\begin{split}
xJ_0&=\sum_{k=0}^\infty(-1)^k\frac{x^{2k+1}}{2^{2k}(k!)^2}\\
J_0'&=\sum_{k=0}^\infty(-1)^{k+1}\frac{x^{2k+1}}{2^{2k+1}k!(k+1)!}\\
xJ_0''&=\sum_{k=0}^\infty(-1)^{k+1}\frac{(2k+1)x^{2k+1}}{2^{2k+1}k!(k+1)!},
\end{split}
\]
so the coefficient of $x^{2k+1}$ in $xJ_0''+J_0'+xJ_0$ is $(-1)^k$ times
\[\frac{1}{2^{2k}(k!)^2}-\frac{1}{2^{2k+1}k!(k+1)!}-\frac{(2k+1)}{2^{2k+1}k!(k+1)!}
=\frac{1}{2^{2k}(k!)^2}\left[1-\frac{1}{2(k+1)}-\frac{2k+1}{2(k+1)}\right]=0.
\]


\end{Solution}
\newpage
{\bf 4.\ }  The purpose of this exercise is to illustrate  Newton's Lemma II in the {\it Principia}, using the function 
\[f(x)=\frac{\sin x}{x}\]
on the interval $[0,\pi]$. 

a) Draw the graph of $f(x)$ and the rectangles illustrating the upper and lower sums when the interval $[0,\pi]$ is divided into four equal parts. 

b) Compute the upper and lower sums, simplifying your answer. What is the difference between these upper and lower sums?

c) Suppose you divided $[0,\pi]$ into $n$ equal parts. Find the difference  between the upper and lower sums without actually computing them individually. How does this relate to what Newton claims in his Lemma II?


\begin{Solution}
a)
\begin{center}
\begin{tikzpicture}[scale=4.0]
   % \fill[fill=blue!20] (0,0) -- plot [domain=0.001:3.14] 
    %(\x,{sin(\x r)/(\x)}) -- (3.14,0)  -- cycle;
    \draw plot[domain=0.01:3.14] (\x,{sin(\x r)/(\x)}) ;
    %\node[label=below: $1$] at (1,0) {};
    %\node[label=below: $x$] at (2,0) {};
    \draw[black,->] (-0.2,0) -- (3+0.3,0) node[right]{$x$};
    \draw[black,->] (0,-0.2) -- (0,1+0.3) node[above]{$y$};
  \node[label=left: $1$ ] at (0,1) {};
  
     \draw[black] (.785,0) -- (.785,1.41/1.57) ;
         \draw[blue] (.785,1.41/1.57) -- (.785,1) ;
         \draw[blue] (0,1) -- (.785,1) ;

         \node[label=below: $\frac{\pi}{4}$ ] at (.785,0) {};

     
    \draw[black] (1.57,0) -- (1.57,1/1.57) ;
     \draw[blue] (1.57,1/1.57) -- (1.57,1.41/1.57) ;
      \draw[blue] (.785,1.41/1.57) -- (1.57,1.41/1.57) ;
       \draw[red] (.785,1.41/1.57) -- (0,1.41/1.57) ;
    \node[label=below: $\frac{\pi}{2}$ ] at (1.57,0) {};
   
    \draw[black] (2.355,0) -- (2.355,1.41/4.71) ;
    \draw[blue] (2.355,1.41/4.71) -- (2.355,1/1.57) ;
     \draw[blue] (2.355,1/1.57) -- (1.57,1/1.57) ;
      \draw[red] (1.57,1/1.57) -- (.785,1/1.57) ;
     
     \node[label=below: $\frac{3\pi}{4}$ ] at (2.355,0) {};
     
    \node[label=below: $\pi$ ] at (3.14,0) {};
     \draw[blue] (3.14,0) -- (3.14,1.41/4.71) ;
      \draw[blue] (3.14,1.41/4.71) -- (2.355,1.41/4.71) ;
       \draw[red] (2.355,1.41/4.71) -- (1.57,1.41/4.71) ;
\end{tikzpicture}
\end{center}

b)

The lower sum is 
\[
L_4=\frac{\pi}{4}\left[ \frac{1/\sqrt 2}{\pi/4}+\frac{1}{\pi/2}+\frac{1/\sqrt{2}}{3\pi/4}+0\right]
=
\frac{1}{2}+\frac{4}{3\sqrt{2}}
\]
and the upper sum is 
\[U_4=\frac{\pi}{4}\left[ 1+\frac{1/\sqrt 2}{\pi/4}+\frac{1}{\pi/2}+\frac{1/\sqrt{2}}{3\pi/4}+\right]=\frac{\pi}{4}+\frac{1}{2}+\frac{4}{3\sqrt{2}}
\]

and 
\[U_4-L_4=\frac{\pi}{4},\]
which is the area of the tallest rectangle, the one over $[0, \pi/4]$. 

c) 

For any $n$, the upper and lower sums $U_n$ and $L_n$ have the same terms except $U_n$ has an extra term from the rectangle over $[0,\pi/n]$, which has area $\pi/n$. 
So 
\[U_n-L_n=\frac{\pi}{n}.\]
Since $\frac{\pi}{n}\to 0$, this confirms Newton's claim in his Lemma II that the upper sums, lower sums, and the area between approach one another to within any given difference, or as he says, ``the ultimate ratios of [these areas] to one another, are ratios of equality". 


\end{Solution}

\end{document}









\end{document}

 
 
