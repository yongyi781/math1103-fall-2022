\documentclass[12pt]{article}  

%Run with XeLaTeX
\usepackage
[colorlinks=true, pdfstartview=FitV, linkcolor=blue, citecolor=blue, urlcolor=blue]
{hyperref}

\usepackage{amssymb}  
\usepackage{amsthm}
\usepackage{amsmath}
\usepackage{graphics} 
\usepackage{graphicx} 
%\usepackage[latin1]{inputenc}
\usepackage{tikz}
\usepackage{pgfplots}
\usepackage{wrapfig}
\usepackage{caption}
\usepgfplotslibrary{polar}
\usepackage{ skull }
\usetikzlibrary{decorations.fractals}
%\usepackage{pst-func}

% GNUPLOT required
\usepackage{verbatim}

\linespread{1.3}

%\addtolength{\textwidth}{80pt}
\addtolength{\evensidemargin}{20pt}
\addtolength{\oddsidemargin}{20pt}

%%%%%%%%%%%%%%%%%%%%%%%%%%%%%%%%%%%%%%%%%%%%%%
%  Begin user defined commands

\newcommand{\map}[1]{\xrightarrow{#1}}


\newcommand{\bz}{\mathbb Z}
\newcommand{\bq}{\mathbb Q}
\newcommand{\br}{\mathbb R}
\newcommand{\bc}{\mathbb C}
\newcommand{\al}{\alpha}
\newcommand{\be}{\beta}
\newcommand{\ga}{\gamma}
\newcommand{\de}{\delta}
\newcommand{\ep}{\epsilon}
\DeclareMathOperator{\lub}{l.u.b.}
%  End user defined commands
%%%%%%%%%%%%%%%%%%%%%%%%%%%%%%%%%%%%%%%%%%%%%%


%%%%%%%%%%%%%%%%%%%%%%%%%%%%%%%%%%%%%%%%%%%%%%
% These establish different environments for stating Theorems, Lemmas, Remarks, etc.

\newtheorem{Thm}{Theorem}
\newtheorem{Prop}[Thm]{Proposition}
\newtheorem{Lem}[Thm]{Lemma}
\newtheorem{Cor}[Thm]{Corollary}

\theoremstyle{definition}
\newtheorem{Def}[Thm]{Definition}

\theoremstyle{remark}
\newtheorem{Rem}[Thm]{Remark}
\newtheorem{Ex}[Thm]{Example}

\theoremstyle{definition}
\newtheorem{Exercise}{Problem}

\newenvironment{Solution}{\noindent\textbf{Solution.}}

%\renewcommand{\labelenumi}{(\alph{enumi})}
\renewcommand\qedsymbol{QED}
% End environments 
%%%%%%%%%%%%%%%%%%%%%%%%%%%%%%%%%%%%%%%%%%%%%%%
%Some commands to save paper


\setlength{\parindent}{0in}
\setlength{\parskip}{8pt}

\DeclareMathOperator{\arcsec}{arcsec}
\DeclareMathOperator{\arccot}{arccot}
\DeclareMathOperator{\arccsc}{arccsc}
\DeclareMathOperator{\LH}{\ \underset{\text{LH}}{=}\ }
\newcommand{\Dep}{\Delta_+}
\newcommand{\Dem}{\Delta_-}
\newcommand{\bu}{\mathbf u}
\newcommand{\bv}{\mathbf v}
\newcommand{\bw}{\mathbf w}

\newcommand{\ora}{\overrightarrow}



\addtolength{\textwidth}{80pt}
\addtolength{\evensidemargin}{-40pt}
\addtolength{\oddsidemargin}{-40pt}
\addtolength{\topmargin}{-80pt}
\addtolength{\textheight}{1.8in}

\setlength{\parindent}{0in}
\setlength{\parskip}{8pt}

\DeclareMathOperator{\arcsinh}{arcsinh}

%%%%%%%%%%%%%%%%%%%%%%%%%%%%%%%%%%%%%%%%%%%%%%
% Now we're ready to start
%%%%%%%%%%%%%%%%%%%%%%%%%%%%%%%%%%%%%%%%%%%%%%

\begin{document}  

%\author{Your Name}
{\bf MATH 1103 Homework 9 }\\
{\bf Due Friday April 13  2018}

{\bf Practice Problems}

{\bf Practice 1.\ } 

a)\ $\int_0^\pi x\sin x\ dx$,\qquad b)\ $\int_0^b xe^{-x}\ dx$ (where $b$ is a constant)
\qquad c) $\int_1^e \log x\ dx$.

\begin{Solution}
These are all integration by parts. 

a)\  $u=x,\ v'=\sin x,\ u'=1,\ v=-\cos x$, so 
\[\int_0^\pi x\sin x\ dx=-x\cos x\big\vert_0^\pi+\int_0^\pi \cos x=\pi.\]

b)\ $u=x,\ v'=e^{-x},\ u'=1,\ v=-e^{-x}$, so 
\[\int_0^b xe^{-x}\ dx=xe^{-x}\big\vert_0^b+\int_0^b e^{-x}\ dx=be^{-b}-e^{-b}+1.\]

c)\ $u=\log x,\ u'=\frac{1}{x},\ v'=1,\ v=x$, so
\[\int_1^e \log x\ dx=x\log x\big\vert_1^e-\int_1^e \frac{1}{x}\cdot x\ dx=e-(e-1)=1.\]
\end{Solution}

{\bf Practice 2.\ } Let $c$ and $r$ be constants

a)\ Find antiderivatives of 
$e^{cx}, \sin(cx), \cos( cx), (cx)^{r}$. 

b)\ If $F(x)$ is an antiderivative of $f(x)$, find an antiderivative of $f(cx)$. 

\begin{Solution}

a) $\dfrac{1}{c}e^{cx},\ -\dfrac{1}{c}\cos(cx),\ \dfrac{c^r}{r+1}x^{r+1}$.

b) $\frac{1}{c}F(cx)$. 

\end{Solution}

{\bf Practice 3.\ } For constants $a,b$ compute the indefinite integrals
\[\int e^{ax}\cos bx\ dx, \qquad \text{and}\qquad \int e^{ax}\sin bx\ dx.\]
[Hint: Integration by parts, and don't give up!]

\begin{Solution} Do integration by parts twice, both times letting $u=e^{ax}$ and letting $v'$ be the trig function. 
\[\begin{split}
\int e^{ax}\cos bx\ dx&=\frac{e^{ax}}{a^2+b^2}\left(b\sin bx+a\cos bx\right)+C\\
\int e^{ax}\sin bx\ dx&=\frac{e^{ax}}{a^2+b^2}\left(a\sin bx-b\cos bx\right)+C
\end{split}\]
\end{Solution} 

{\bf Practice 4.\ } Let $f$ be a polynomial.

a) Show that 
\[\int_0^\infty f(x) e^{-x}\ dx=f(0)+\int_0^\infty f'(x) e^{-x}\ dx.\]
(Here, $\int_0^\infty$ is defined to be $\lim\limits_{b\to\infty}\int_0^b$.)

b)\ Find a formula for $\int_0^\infty f(x) e^{-x}\ dx$ in terms of the derivatives $f^{(k)}(0)$. 

c)\ Use your formula to show that $\int_0^\infty x^n e^{-x}\ dx=n!$.


\begin{Solution} a)\ $u=f$, $v'=e^{-x}$, $u'=f'$, $v=-e^{-x}$.
$uv\big\vert_0^b=-f(b)e^{-b}+f(0)\to f(0)$ because exponentials crush polynomials.

b)\ Replacing $f$ by $f'$ then $f''$, etc, we get
\[\int_0^\infty f(x) e^{-x}\ dx
=\sum_{k=0}^nf^{(k)}(0),
\] where $n$ is the degree of the polynomial $f$. 

c)\ For $f(x)=x^n$ we have $f^{(n)}(0)=n!$ and $f^{(k)}(0)=0$ for $0\leq k<n$. 
\end{Solution}

{\bf Practice 5.\ } Let $n$ be a positive integer. Use integration by parts to show that  
\[\int_0^{\pi/2}\cos^{n+1}(\theta)\ d\theta=\frac{n}{n+1}\int_0^{\pi/2}\cos^{n-1}(\theta)\ d\theta.
\]
and use this to compute $\int_0^{\pi/2}\cos^{n}(\theta)\ d\theta$ for any integer $n$. 

\begin{Solution} See analogous (in fact the same) solution for $\sin^n\theta$ in section 9.8.1 in the notes. 
\end{Solution}

{\bf Practice 6.\ } Let $p$ and $q$ be integers, with $p>0$ and $q\geq 0$. 
Consider the integral
\[W_{p,q}=\displaystyle{\int_0^1(1-x^{1/p})^q\ dx.}\]
a)\ Using integration by parts, show that if $q\geq 1$ then 
$\displaystyle{W_{p,q}=\frac{q}{p+q} W_{p,q-1}.}$
\newline
b)\ Using part a), show that 
$\displaystyle{\int_0^1(1-x^{1/p})^q\ dx=\frac{p!\ q!}{(p+q)!}}$

\begin{Solution}
Integrating by parts with 
\[ u=(1-x^{1/p})^q,\quad v'=1,\quad\text{so}\quad 
u'=-\frac{q}{p}(1-x^{1/p})^{q-1}\cdot x^{-1+1/p},
\quad v=x.\]
Since $uv\big\vert_0^1=0$, we get
\[\begin{split}
W_{p,q}&=\frac{q}{p}\int_0^1(1-x^{1/p})^{q-1}\cdot x^{-1+1/p}\cdot x\ dx\\
&=\frac{q}{p}\int_0^1(1-x^{1/p})^{q-1}\cdot \left(x^{1/p}\right)\ dx\\
&=\frac{q}{p}\int_0^1(1-x^{1/p})^{q-1}\cdot \left(1-(1-x^{1/p})\right)\ dx\\
&=\frac{q}{p}\int_0^1\left[(1-x^{1/p})^{q-1}-(1-x^{1/p})^{q}\right]\ dx\\
&=\frac{q}{p}\left(W_{p,q-1}-W_{p,q}\right),
\end{split}\]
so
\[
\left(1+\frac{q}{p}\right) W_{p,q}=\frac{q}{p}\cdot W_{p,q-1}.
\]
Solving for $W_{p,q}$ we get 
\[W_{p,q}=\frac{q}{p+q}\cdot W_{p,q-1}\]
as desired.
Now we have
\[W_{p,q}=\frac{q}{p+q}\cdot W_{p,q-1}=\frac{q}{p+q}\cdot\frac{q-1}{p+q-1}\cdot W_{p,q-2}
=\cdots
\frac{q}{p+q}\cdot\frac{q-1}{p+q-1}\cdots\frac{1}{p+1}\cdot W_{p,0}
=\frac{q!\ p!}{(p+q)!}
\]

\end{Solution}
\vskip10pt
Problems to turn in are on the next page
\newpage

{\bf Homework 9 problems to be turned in.}

The goal of this entire assignment is to prove that $\pi$ is an irrational number. In fact you will prove the stronger statement that $\pi^2$ is irrational. Practice problem 4 is a good warmup.


{\bf 1.\ } Suppose $f$ is a polynomial function. Use integration by parts to prove that 
\[\int_0^1 f(x)\sin(\pi x)\ dx=\frac{f(0)+f(1)}{\pi}-\frac{1}{\pi^2}\int_0^1f''(x)\sin(\pi x)\ dx.
\]
\begin{Solution}
Taking $u=f$ and $v'=\sin(\pi x)$ we have $u'=f'$ and $v=-\frac{1}{\pi}\cos(\pi x)$, so 
\[\begin{split}\int_0^1 f(x)\sin(\pi x)\ dx
&=-\frac{f(x)}{\pi}\cos(\pi x)\Big\vert_0^1+\frac{1}{\pi}\int_0^1 f'(x) \cos(\pi x)\\
&=\frac{f(0)+f(1)}{\pi}+\frac{1}{\pi}\int_0^1 f'(x) \cos(\pi x).
\end{split}
\]
Now taking $u=f'$ and $v'=\cos(\pi x)$ we have $u'=f''$ and $ v=\frac{1}{\pi}\sin(\pi x)$, so 
\[\begin{split}
\int_0^1 f(x)\sin(\pi x)\ dx
&=\frac{f(0)+f(1)}{\pi}+\frac{1}{\pi}
\left[\frac{f'(x)}{\pi}\sin(\pi x)\Big\vert_0^1-\frac{1}{\pi}\int_0^1 f''(x) \sin(\pi x)\right]\\
&=\frac{f(0)+f(1)}{\pi}-\frac{1}{\pi^2}\int_0^1 f''(x) \sin(\pi x),
\end{split}
\]
since $\sin(0)=\sin(\pi)=0$.  
\end{Solution}

{\bf 2.\ } Suppose $f$ is a polynomial  of even degree $2n$.   Use the formula in Problem 1 to prove that 
\[\int_0^1 f(x)\sin(\pi x)=\sum_{k=0}^{n}(-1)^k\frac{f^{(2k)}(0)+f^{(2k)}(1)}{\pi^{2k+1}}.\]
(Here $f^{(2k)}$ is the $2k$-th derivative of $f$.)

\begin{Solution} Replacing $f$ by $f''$ in exercise 1 we get 
\[\int_0^1 f''(x)\sin(\pi x)=\frac{f''(0)+f''(1)}{\pi}-\frac{1}{\pi^2}\int_0^1f^{(4)}(x)\sin(\pi x), 
\]
so
\[\int_0^1 f(x)\sin(\pi x)=
\frac{f(0)+f(1)}{\pi}-\frac{f''(0)+f''(1)}{\pi^3}+\frac{1}{\pi^4}\int_0^1f^{(4)}(x)\sin(\pi x).
\]
Doing this $n$ times we get 
\[\int_0^1 f(x)\sin(\pi x)=\sum_{k=0}^{n}(-1)^k\frac{f^{(2k)}(0)+f^{(2k)}(1)}{\pi^{2k+1}},
\]
since $f^{(2n+2)}=0$. 
\end{Solution}

{\bf 3.\ } Fix a positive integer $n$ and let 
$f(x)=\dfrac{x^n(1-x)^n}{n!}$. 
Prove the following. 
\begin{itemize}
\item[a)] 
$f^{(2k)}(0)=f^{(2k)}(1)$. [Hint: How is $f(x)$ related to $f(1-x)$?]
\item[b)]  All derivatives of $f(x)$ take integer values at $x=0$. 
[Hint: How is $f^{(m)}(0)$ related to the coefficient of $x^{m}$ in $f(x)$?]
\item[c)] We have $f(x)\leq \dfrac{1}{n!}$ for all $0\leq x\leq 1$. [Hint: this is just good old differential calculus.]
\end{itemize}

\begin{Solution} a)\ First, we have $f(x)=f(1-x)$ so $f'(x)=-f'(1-x)$, $f''(x)=f''(1-x)$ and in general $f^{(2k)}(x)=f^{(2k)}(1-x)$. So $f^{(2k)}(0)=f^{(2k)}(1)$. 

b)\ For any polynomial $p(x)$, $p^{(m)}(0)$ is $m!$ times the coefficient of $x^m$ in $p(x)$. The polynomial 
\[x^n(1-x)^n=\sum_{m=n}^{2n}c_m x^m\]
where the coefficients $c_m$ are integers. So we have
$f^{(m)}(0)=0$ if $m<n$ or $m>2n $ and 
$f^{(m)}(0)=\dfrac{m!}{n!}c_m$
if $n\leq m\leq 2n$. In this last case $m!/n!$ is an integer. So for all $m$ we have that $f^{(m)}(0)$ is an integer.

c)\ Writing $f$ as 
\[f(x)=\frac{1}{n!}(x-x^2)^n\]
we have 
\[f'(x)=\frac{n}{n!}(1-2x)(x-x^2)^{n-1}.\]
This is zero only at $x=0,\frac{1}{2}, 1$. Since $f(0)=f(1)=0$ the maximum value of $f$ is 
\[f(\tfrac{1}{2})=\frac{1}{4^n n!}\leq \frac{1}{n!}.\]
Actually it is not necessary to work this hard. Just note that $x(1-x)\leq 1$. 

\end{Solution}

\newpage
\rule{\textwidth}{1pt}

Now assume $\pi^2$ is a rational number, say $\pi^2=\dfrac{a}{b}$ where $a$ and $b$ are integers. Since $\lim_{n\to\infty}\dfrac{a^{n}}{n!}=0$, we can choose a positive integer $n$ so that $\dfrac{a^{n}}{n!}<\frac{1}{2}$. 
With this choice of $n$ let $f(x)$ be the  polynomial 
$f(x)=\dfrac{x^n(1-x)^n}{n!}$ from Problem 3, and let 
\[N= \pi a^{n}\int_0^1 f(x)\sin(\pi x).\]

\rule{\textwidth}{1pt}

\vskip10pt
{\bf 4.\ } Use the results of problems 1, 3a) and 3b) to prove that $N$
is an integer. 

\begin{Solution} From problems 1 and 3a)we have
\[\begin{split}
N&=\pi a^n\int_0^1 f(x)\sin(\pi x)\ dx\\
&=\pi a^n\sum_{k=0}^n(-1)^k\frac{f^{(2k)}(0)+f^{(2k)}(1)}{\pi^{2k+1}}\\
&= 2a^n\sum_{k=0}^n(-1)^kf^{(2k)}(0)\cdot \frac{1}{\pi^{2k}}\\
&=2a^n\sum_{k=0}^n(-1)^kf^{(2k)}(0)\cdot\frac{b^k}{a^k}\\
&=2\sum_{k=0}^n(-1)^kf^{(2k)}(0)\cdot a^{n-k}b^k.
\end{split}
\]
This is an integer by 3b). 
\end{Solution}

\vskip10pt
{\bf 5.\ } Prove that
$0<N<1$
and explain why this proves $\pi^2$ is irrational. 


\begin{Solution} The integral $\int_0^1 f(x)\sin(\pi x)\ dx$ is positive because $f(x)\sin(\pi x)>0$ on $(0,1)$. So $0<N$.  From 3c we have 
\[\pi a^{n}\int_0^1 f(x)\sin(\pi x)\ dx\leq\pi a^{n}\frac{1}{n!}\int_0^1 \sin(\pi x)\ dx
=\pi a^{n}\frac{1}{n!}\frac{2}{\pi}
=\frac{2a^n}{n!}<1
\] 
by our choice of $n$. Thus, based on the assumption that $\pi^2$ is rational we have found a number $N$ which is both an integer yet lies in $(0,1)$. this is a contradiction, so $\pi^2$ must be irrational. 


\end{Solution}



\end{document}