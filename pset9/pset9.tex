\documentclass[11pt,oneside]{amsart}
\usepackage{geometry}
\usepackage{amssymb,parskip,mathtools}
\usepackage[shortlabels]{enumitem}
\usepackage[colorlinks]{hyperref}

\theoremstyle{definition}
\newtheorem{problem}{Problem}

\theoremstyle{plain}
\newtheorem{theorem}{Theorem}

\newcommand{\bC}{\mathbb{C}}
\newcommand{\bQ}{\mathbb{Q}}
\newcommand{\bR}{\mathbb{R}}
\newcommand{\bZ}{\mathbb{Z}}
\newcommand{\bE}{\mathbb{E}}
\newcommand{\eps}{\varepsilon}
\newcommand{\blank}{\underline{\hspace{1cm}}}
\newcommand{\longblank}{\underline{\hspace{2cm}}}

\DeclareMathOperator{\Var}{Var}

\title{MATH1103 Fall 2022\\
Problem Set 9}

\begin{document}
\maketitle
This problem set is due on Wednesday, November 9 at 11:59 pm. Each problem part is worth 3 points. Collaboration is encouraged. In all cases, you must write your own solutions, and and you must cite collaborators and resources used.

\begin{problem}
Binary numbers are the bread and butter of how computers calculate things. What is the exact value of this binary number?
\[x=0.01010101\ldots.\]
If you are unfamiliar with binary, the $k$th position after the decimal point is the $\frac 1{2^k}$-place value. So for example, a finite binary expansion such as $0.01101$ would be equal to $\frac 14+\frac 18+\frac 1{32}$.
\end{problem}

\begin{problem}
A king once lost in a game of chess to a traveller, and offered the traveller a prize of his choice. The traveller said:
\begin{quote}
  I am a modest man, so I will only request this: on this chessboard in front of us, put one grain of rice on the first square, two grains of rice on the second square, four on the third square, and so on, doubling each time until the 64th square.
\end{quote}

The king laughed and said, ``That's all? You are too modest.''

Do you agree with the king that the traveller is too modest? How many grains of rice did the traveller request? What fraction of the total rice belongs to the last square?
\end{problem}

\begin{problem}
Determine whether the following series converge or diverge, and find the sum of those that converge.
\begin{enumerate}[(a)]
  \item \[\frac 1{10000}+\frac 1{10001}+\frac 1{10002}+\frac 1{10003}+\cdots\]
  \item \[\frac 1{10000}+\frac 1{20000}+\frac 1{30000}+\frac 1{40000}+\cdots\]
  \item \[\frac 1{10000}+\frac 1{20000}+\frac 1{40000}+\frac 1{80000}+\cdots\]
\end{enumerate}
\end{problem}

\begin{problem}
In this problem you will uncover many different ways to sum the series
\[1+2x+3x^2+4x^3+\cdots=\sum_{n=0}^\infty (n+1)x^n.\]
Spoiler alert: we'll find that the series sums to $1/(1-x)^2$ (whenever it converges).
\begin{enumerate}[(a)]
  \item For now, assume $x$ is any number that makes the series converge. Recall the method mentioned in Problem Set 5 Problem 3(d) (Strang 8.4.20)! If necessary, relearning the method is part of this problem! Then work out the sum using that method.
  \item Here is a completely different approach. First notice that the $k$th term above is $kx^{k-1}$ which is precisely the derivative of $x^k$. So
        \[1+2x+3x^2+4x^3+\cdots\]
        is precisely the derivative of
        \[1+x+x^2+x^3+\cdots.\]
        This is, of course, the geometric series we all know and love. You know what the geometric series sums to\ldots perhaps you can take the derivative of both sides of the geometric series formula? See where this leads.
  \item Notice that $1/(1-x)^2$ is the square of $1/(1-x)$. Therefore, it somehow must be true that
        \[(1+x+x^2+\dots)^2=1+2x+3x^2+4x^3+\dots.\]
        See if you can argue why this equation is true by going through the expanding process on the left hand side. For example, I can see that the coefficient of $x$ in $(1+x+x^2+\dots)^2$ should be 2 because when we multiply $(1+x+x^2+\dots)(1+x+x^2\dots)$, the only two ways to get an $x^1$ term are to pick 1 from the first group and $x$ from the second group, or to pick $x$ from the first group and 1 from the second group.
  \item What series do you think equals $1/(1-x)^3$? (Any reasonable guess with some explanation of why you think it's true suffices.)
  \item (Optional) Let's address convergence. Find a formula for the \emph{finite} sum
        \[1+2x+3x^2+4x^3+\dots+(n+1)x^n\]
        (notice I'm stopping at $n$ here), and use what you know about growth rates to find the values of $x$ for which the series converges.
\end{enumerate}
\end{problem}

\begin{problem}
Rate the difficulty of each problem (1, 2, 3a, 3b, 3c, 4a, 4b, 4c, 4d) according to the following scale. Your ratings will collectively let me know which areas are difficult in this class. Thanks for your feedback!
\begin{itemize}
  \item 1 -- Super easy, barely an inconvenience!
  \item 2 -- Not easy, but I was able to solve the problem on my own by comparing it with an example from class or the textbook.
  \item 3 -- Not easy, but I was able to solve the problem on my own through observations, analysis, and/or creative reasoning.
  \item 4 -- I made some progress but got stuck, and with help, I was able to solve the problem. I feel like I understand it now.
  \item 5 -- I could not start this problem without help, but after getting help I was able to solve the problem. I feel like I understand it now.
  \item 6 -- I could not start this problem without help, but after getting help I was able to solve the problem. However, I still don't feel like I understand what is going on in this problem.
  \item 7 -- I could not solve the problem, even with help.
\end{itemize}
\end{problem}
\end{document}
