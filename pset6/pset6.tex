\documentclass[11pt,oneside]{amsart}
\usepackage{geometry}
\usepackage{amssymb,parskip,mathtools}
\usepackage[shortlabels]{enumitem}
\usepackage[colorlinks]{hyperref}

\theoremstyle{definition}
\newtheorem{problem}{Problem}

\newcommand{\bC}{\mathbb{C}}
\newcommand{\bQ}{\mathbb{Q}}
\newcommand{\bR}{\mathbb{R}}
\newcommand{\bZ}{\mathbb{Z}}
\newcommand{\bE}{\mathbb{E}}

\DeclareMathOperator{\Var}{Var}

\title{MATH1103 Fall 2022\\
Problem Set 6}

\begin{document}
    \maketitle
    This problem set is due on Wednesday, October 12 at 11:59 pm. Each problem part is worth 3 points. Collaboration is encouraged. In all cases, you must write your own solutions, and and you must cite collaborators and resources used.

    \begin{problem}
        Some exercises.
        \begin{enumerate}[(a)]
            \item (Strang 8.4.3) Why is $p(x)=e^{-2x}$ not an acceptable probability density for $x\geq 0$? Why is $p(x)=4e^{-2x}-e^{-x}$ (also for $x\geq 0$) not acceptable?
            \item (Strang 8.4.5) If $p(x)=e^{-x}$ for $x\geq 0$, find the probability that $X\geq 2$ and the approximate probability that $1\leq X\leq 1.01$.
            \item (Strang 8.4.12) Find the mean of the distribution given by the PDF $p(x)=e^{-x}$ for $x\geq 0$. (Integrate by parts.)
            \item (Strang 8.4.19) Supernovas are expected about every 100 years. What is the probability that you will be alive for the next one?
            
            \emph{Hint}: You can use a Poisson model with $\lambda=0.01\times\text{(your lifetime)}$ and estimate your lifetime. You can also use an exponential distribution.
            
            (Supernovas actually occurred in 1054 (Crab nebula), 1572, 1604, and 1987. But the future distribution doesn't depend on the date of the last one.)
            
            \emph{Remark}: Strang discusses the Poisson distribution on pages 331--332, including in the table at the end of page 332.
            \item (Strang 8.4.24) What is the variance of the uniform distribution on $[0,1]$? What is the standard deviation?
            \item (Strang 8.4.33) Suppose grades have a normal distribution with mean 70 and standard deviation 10. If 300 students take the test and passing is 55, how many are expected to fail? Write your answer as a definite integral then give an approximation. What passing grade will fail 1/10 of the class?
        \end{enumerate}
    \end{problem}

    \begin{problem}
        You know from class that $\Var(X)$ is defined as $\bE[(X-\mu)^2]$, where $\mu=\bE[X]$. Use calculus to prove the following famous identity for variance:
        \[\Var(X)=\bE[X^2]-\bE[X]^2.\]
        \emph{Hint}: If you are stuck, Exercise 8.4.23 in Strang has some spoilers to get you started, but you still need to explain the step Strang asks to explain.
    \end{problem}
    
    \begin{problem}
        Two random numbers $X$ and $Y$ are drawn (independently) from a uniform distribution on $[0,1]$. Find $\bE[\max(X,Y)]$.

        \emph{Hint}: This is a pretty fun problem. Start by finding the CDF of $\max(X,Y)$. Use the fact that $\max(X,Y)\leq z$ if and only if both $X\leq z$ and $Y\leq z$.

        \emph{Optional challenge}: Generalize to the $\max(X_1,\dots,X_n)$ where each $X_i$ is drawn from a uniform distribution on $[0,1]$.
    \end{problem}

    % \begin{problem}
    %     Two random numbers $X$ and $Y$ are drawn from a uniform distribution on $[0,1]$. Let $Z=X+Y$. What is the probability density function of $Z$? In order to figure this out, you should first find the cumulative distribution function of $Z$. Finally, compare the shape of the PDF you obtained with the shape of the PDF of the sum of two dice.

    %     \emph{Hint:} To figure out the cumulative distribution function $\mathrm{CDF}_Z(z)$, i.e.\ the probability that $X+Y$ is less than or equal to $z$, first assume $X=x$, then figure out the probability that $x+Y\leq z$. Then average over $x$ from 0 to 1. Another way to approach this is to draw a square representing the possible values of $X$ and $Y$ together, then for each $z$, finding the area of the region corresponding to the points $(X,Y)$ where $X+Y\leq z$.
    % \end{problem}

    \begin{problem}[Optional]
        \href{https://www.youtube.com/watch?v=4y_nmpv-9lI}{This video} talks about the surprisingly non-intuitive problem of generating a random point inside a circle. After watching the video, think about the following.
        \begin{enumerate}[(a)]
            \item What do you think is the fairest way to pick a random point on a circle?
            \item The video mentioned something called inverse transform sampling. To test if you understood it, what is the probability density function of $X^2$ if $X$ is drawn from a uniform distribution on $[0,1]$? (Recall we saw in class a while ago that the average value of $X^2$ was 1/3.)
            
            \emph{Hint}: As usual, start by finding the CDF.

            \emph{Spoiler}: the PDF of $X^2$ is $\frac 1{2\sqrt x}$ for $0\leq x\leq 1$.
        \end{enumerate}
    \end{problem}

    \begin{problem}[Optional]
        \leavevmode\begin{enumerate}[(a)]
            \item Let $X$ and $Y$ be two independent random variables with PDFs $p(x)$ and $q(x)$ respectively. Prove that the PDF of $X+Y$ is given by
            \[\mathrm{PDF}_{X+Y}(x)=\int_{-\infty}^\infty p(t)q(x-t)\,dt.\]
            \emph{Hint:} Find the CDF of $X+Y$, then differentiate under the integral sign.
            \item We write $X\sim N(\mu,\sigma)$ to mean that $X$ follows a normal distribution with mean $\mu$ and standard deviation $\sigma$. Using part (a), prove that if $X\sim N(0,1)$ and $Y\sim N(0,1)$, then $X+Y\sim N(0,\sqrt 2)$.
        \end{enumerate}
    \end{problem}
\end{document}
